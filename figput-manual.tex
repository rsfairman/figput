
\documentclass[10pt]{article}

% Allows various tweaks to itemize as optional arguments.
\usepackage{enumitem}

% Gives the proof environment, with a box for QED.
\usepackage{amsthm}

% Needed for things like pmatrix.
\usepackage{amsmath}

% This allows footnotes from within array environment.
\usepackage{footnote}
\makesavenoteenv{array}

% To adjust font sizes more easily.
\usepackage{relsize}

% This allows \verb in footnotes
\usepackage{fancyvrb}

% This allows me to say 
% \begin{figure}[H]
% to mean ``here, and don't argue.'' Note the captial 'H'.
\usepackage{float}

% This is the modern thing to do, but the manual is 1300 pages!
\usepackage{tikz}
\usepackage{pgfplots}


\begin{document}

% Need this for fancyvrb to work.
\VerbatimFootnotes

% So that I don't have to type this all the time.
\newcommand{\figput}{\textsc{FigPut}}


\begin{center}
  {\Large \figput}
  
  Interactive Figures for \LaTeX

  \verb=https://github.com/rsfairman/figput=

  Randall Fairman

  Version 0.90, July 21, 2022
\end{center}

\vskip 0.2cm

The \textsc{FigPut} system allows \LaTeX\ to produce interactive
figures. The system produces a static PDF file, as usual; in addition,
the document can be viewed with an Internet browser and the figures
become interactive. See the prototype in {\tt example/example.tex}
to clarify what follows.\footnote{The version available on CTAN
includes only what is strictly necessary to rebuild the document from
scratch: {\tt example.tex} and {\tt exteernalcode.js}. The repository at {\tt
  github} includes all the files that result from the build process,
and are necessary to view the PDF in  either static or interactive
form.} To view this example in its fully interactive form, see
\verb=rsfairman.github.io/figput-example/=.

% For some reason \textsc doesn't work within \section
\section{Why \normalfont\scshape{FigPut}}

There are many ways to produce figures in \LaTeX: MetaPost, TikZ,
PSTricks, XY-pic, {\tt graphics}, {\tt picture}, and more. Creating
technical figures is often an exercise in programming, and none of
these is based on a commonly used programming
language. \textsc{FigPut} is based on JavaScript (JS), which is
probably the most widely used language today. Moreover,
\figput\ allows figures to be animated and/or interactive.

Aside from the fact that figures are specified with
JavaScript, the system is normal \LaTeX. An important feature is that
the two forms of the document -- static PDF and interactive -- look 
identical, with the same pagination and formatting. The reader can
move between the printed version and the interactive version without
needing to reorient. 

% For some reason \textsc doesn't work within \section
\section{Using \normalfont\scshape{FigPut}}

Here's a list of all the commands available in \figput:
\begin{verbatim}
\begin{figput} ... \end{figput}
\FigPut
\SetInnerMargin
\SetOuterMargin
\NeverSkip
\AllowSkip
\LoadFigureCode
\end{verbatim}

\subsection{The \LaTeX\ Side}

Interactive figures are specified with the {\tt figput}
environment or the \verb=\FigPut= command. The full form of the
environment specification is
\vskip 0.2cm

\noindent\verb=\begin{figput}{<fig_name>,<fig_ht>}[=\emph{optional arguments}\verb=]=

\noindent\verb=  <JavaScript code>=

\noindent\verb=\end{figput}=
\vskip 0.2cm

\noindent There are two mandatory arguments: \verb=<fig_name>= and
\verb=<fig_ht>=. Every figure must have a unique name. This name is
unrelated to any \verb=\label= or other identifier used for
the figure. The \verb=<fig_ht>= is the height of the figure, expressed
like other \LaTeX\ lengths in {\tt pt}, {\tt cm}, \emph{etc}. The
figure is allocated this much vertical space on the page. The
remaining optional arguments are discussed below.

The {\tt <JavaScript code>} is used to draw the figure, and the only
requirement is that it be given as
\begin{verbatim}
function <fig_name>(ctx) {
  ...
}
\end{verbatim}
The name of the JS function must match the figure name given to the
{\tt figput} environment -- that's how the browser finds the correct
JS function to draw a particular figure. The {\tt ctx} argument
is required; it's the ``rendering context'' used for
drawing.\footnote{\label{page-ctx} A variable of this type is commonly obtained in
JavaScript by something like \vspace{-1.5mm}
\begin{verbatim}
let canvas = document.getElementById('where-drawing-happens');
let ctx = canvas.getContext('2d');
\end{verbatim}
\vspace{-1.5mm} The full name for this type is {\tt
  CanvasRenderingContext2D}.}

All of the JS functions for the various figures share the same
name-space, and it is perfectly acceptable to share functions between
figures or to define helper functions within a figure's
environment. If many figures use the same {\tt drawDoodad()}
function, then that function should appear with only a single figure, and
it will be available to all of them. 

Use \verb=\SetInnerMargin= and \verb=\SetOuterMargin= to set the inner and outer
margins (which may differ in a book that has left and right
pages). The figure will be rendered within a rectangle that is set in
from the left margin by the given amount. For example, you might say
\begin{verbatim}
\SetInnerMargin{150pt}
\SetOuterMargin{150pt}
\end{verbatim}
so that the left edge of the figure is {\tt 150pt} from the left edge
of the paper. These values can be changed before every figure, if that
is appropriate.

The coordinate system used within the JS code is based on big points ({\tt
  bp}), with the origin at the lower-left corner of the
figure. The $y$-coordinate increases as you move \emph{up} the page.
The $y$-axis is offset from the edge of the page by the
value given to \verb=\SetInnerMargin= or \verb=\SetOuterMargin=.

There is no need to begin the drawing by erasing the drawing
area. That's done automatically.

\subsubsection*{Optional Arguments}

The arguments to {\tt figput} in {\tt []}, are optional. The first two
of these arguments are \label{ht-above-below}
\begin{verbatim}
ht_above, ht_below 
\end{verbatim}
If these arguments appear, then they must both appear, they must be
the first two arguments, and they must be given in that order (above,
then below); furthermore, they must include a unit specification, like {\tt
  bp}. An interactive figure may need a bit of extra clearance 
for an animation or space for widgets that aren't part of the
figure itself. The static version of the document always uses
\verb=<fig_ht>= for the figure height, while the interactive form adds
this optional amount of space above and below the figure. If
\verb=ht_below= is non-zero, then the $y$-axis is \verb=ht_below=
above the bottom of the figure when drawn interactively.

The remaining optional arguments may appear in any order and function
as boolean flags:

\begin{itemize}[itemsep=-1pt,leftmargin=2cm]
  \item[{\tt nostatic}] Use this when the figure is to be shown only
    when the document is viewed interactively. No figure will appear
    in the PDF. Almost certainly, the mandatory \verb=fig_ht= argument
    should be zero. The sum of \verb=ht_above= and \verb=ht_below= is
    used for the height of the interactive figure.
  \item[{\tt done}] indicates to the browser that the {\tt .tikz}
    should not be updated.
  \item[{\tt skip}] TikZ files can be large and take a long time to
    process. When this is set, the TikZ file won't be loaded and the
    figure will appear with a default ``not available'' message.
\end{itemize}

When the composition of a figure is complete, it often makes sense to
turn on the {\tt done} and {\tt skip} flags. Using {\tt skip} saves
time, and {\tt done} prevents inadvertent overwriting of a {\tt .tikz}
file that you're happy with. However, to see the complete output, with
your figures, you would have to go back and tediously remove all of
the {\tt skip} flags. Use \verb=\NeverSkip= to avoid that tedium and
disable all the {\tt skip} 
flags that occur until the next \verb=\AllowSkip=. 

\subsection*{The {\tt \textbackslash FigPut} Command}

\verb=\FigPut= takes exactly the same arguments as the {\tt figput}
environment, but the JS code must appear in an external file. The
location of this external file is indicated with
\verb=\LoadFigureCode=. If the code is found in {\tt mydrawcode.js},
then
\begin{verbatim}
\LoadFigureCode{mydrawcode.js}
\end{verbatim}
must appear somewhere in the {\tt .tex} file.

It is often easier to use an external file to consolidate the drawing
code in fewer files (or a single file). Not only
does it make the {\tt .tex} file shorter and easier to navigate, but it
allows the use of programming tools tailored to JS. For instance,
the code could be written in TypeScript and compiled to JS. On the other
hand, if the code for the figures is brief, then it may be clearer and more
direct to use the {\tt figput} environment so that the code is in the
{\tt .tex} file. 

\subsubsection*{Additional Remarks}

There is no explicit provision in \textsc{FigPut} for changing the
body of the text based on whether the document is to be viewed statically
or interactively. This is intentional. The aim is for the text and its
layout to be \emph{identical} in the two scenarios. However, it would
be relatively straightforward to set up a couple of commands like
\begin{verbatim}
\newcommand{\statictext}[1]{#1}
\newcommand{\intereactivetext}[1]{ }
\end{verbatim}
In the above example, anything that appears in \verb=\statictext{}=
will pass through to the \LaTeX\ document, while everything in
\verb=\interactivetext{}= will be ignored. Swap these two definitions
depending on whether the document is being compiled for interactive or
static reading.

\subsection{Workflow}

There are several phases to composing and releasing a document. The
first phase is a loop through writing the \LaTeX, compiling it, viewing
the output in a browser and generating TikZ, until the document is
done. Everything in the first phase is local to your machine. The
second phase is posting the result to a public-facing website. The
first phase is discussed here; see Section \ref{sec-distributing} for
the second phase.

\subsubsection*{Setup}

The functionality of \figput\ is given by {\tt figput.sty}. It
requires the following packages: {\tt zref}, {\tt xsim}, {\tt tikz}
and {\tt verbatim}.

A typical arrangement is to have a {\tt javascript}
directory with the files for the browser side of the framework, and a
\LaTeX\ directory where the document is written in the usual way. The {\tt 
  javascript} directory contains the JS files and the one HTML file needed
to provide the browser front-end, and you'll also find {\tt server.py}.
This Python script runs a local server so that your browser can
open and view the document being written.\footnote{This requires
Python 3 -- although that's probably obvious these days.} In principle, it's no
different than a normal web-server, but it is tailored in several
ways to the task at hand. To invoke this server, use the command-line
to go to the {\tt javascript} directory and type
\begin{verbatim}
python server.py directory/nameof.pdf [port_num]
\end{verbatim}
The {\tt directory/nameof.pdf} argument is required and should be the
path from the current ({\tt javascript}) directory to the PDF being
generated by \LaTeX. Often, this will be something like
\begin{verbatim}
../../latex_area/num_theory/hairy_math.pdf
\end{verbatim}
The optional \verb=port_num= argument indicates the TCP/IP port on which the
server will listen. It defaults to 8000. In principle, any integer in
the range $[0,2^{16})$ could appear here, but certain values are
restricted and others may already be in use. Stick to values
in the range 8000-8100, and you should be safe.\footnote{See
\verb=en.wikipedia.org/wiki/List_of_TCP_and_UDP_port_numbers=}

By using different port numbers, it's possible to run several servers
at once, so that multiple documents can be viewed and edited at the
same time. Once the server is running, point your browser to
\verb=localhost:8000= (or whatever \verb=port_num= you've chosen) to
load the document.

As stated, {\tt server.py} differs from a normal web-server in certain
ways. This is done to make it easier to work with a document while it
is being written. Once the document is complete, it can be served by a
normal web-server.

\subsubsection*{Additional Files}

There are two additional directories in the {\tt javascript} directory:
{\tt development} and {\tt release}. \figput\ was written in
TypeScript, and the {\tt .ts} files are in {\tt development}. Even if
you don't want to modify the system, the TypeScript files are easier
to follow than the JavaScript files, and are the first place to look to
clarify how the system works internally. There are four files:
\begin{itemize}[itemsep=-1pt,leftmargin=1cm]
\item{\tt main.ts} is the main entry-point for the program. This is
  unlikely to interest anyone who wants to simply use \figput.
\item{\tt layout.ts} handles page layout -- also uninteresting to most people.
\item{\tt widgets.ts} has the code for the various widgets. If widgets
  aren't acting the way you expect, then look here.
\item{\tt tikz.ts} handles lower-level geometry, and conversion from
  JS to TikZ. Look here for details about the {\tt Point2D} and
  {\tt FPath} classes.
\end{itemize}

Two files found in the {\tt javascript} directory are {\tt
  pdf.worker.min.js} and {\tt pdf.js}. These are the source files for
Mozilla's pdf.js library. The complete project can be found at
\verb=github.com/mozilla/pdf.js=. The files are included here because,
when developing a \LaTeX\ document on your local machine, it would be
pointless (and slow) to download these files every time an updated
version of your document is loaded.

The {\tt release} directory contains the files to be used when the
document is released to a public-facing website. As noted above, {\tt
  server.py} works differently than a normal web-server. The release
process is discussed in more detail in Section \ref{sec-distributing}.

\subsubsection*{From JavaScript to TikZ}

Once the server is running, a browser can view documents created with
\figput, but the figures will not be visible in
the PDF document until a TikZ file is provided. The \emph{Get TikZ}
button in the browser window generates the TikZ files. The server
will save these files to the same directory as the PDF file. Note that
a TikZ file will only be generated if the figure has been scrolled
into view (or nearly into view) since the document was loaded.

Because TikZ files can take time to load, it's often useful to
compose the figures one (or a few) at a time. Apply the optional
{\tt done} argument for any figures that are complete, and the
\emph{Get TikZ} button will not generate TikZ files for them. You may
want a particular frame from an animation; once you have the right
frame, it would be frustrating to accidentally overwrite it in the course of 
working on some other figure. Using {\tt done} prevents that from happening.

\subsection{JavaScript Drawing}

In most respects, any JS code can be used with FigPut, but it must be
possible to translate the JS drawing commands to TikZ, and that leads
to certain requirements. In a nutshell, all drawing must be done using a
class that is very similar to JS's built-in {\tt Path2D} class. 

If this is too restrictive, then it is possible to side-step the
requirement with a bit of extra effort. If you use JS drawing commands
that are off-limits, like {\tt drawImage()}, then the interactive version
will work fine, but you'll have to find some other way to generate the
figure for inclusion in the static PDF. Any of the usual methods of
generating figures for \LaTeX\ documents will work, but now the two
versions, interactive and static, are generated from different
source-code.

Earlier, in a footnote on p. \pageref{page-ctx}, it was a stated that
the {\tt ctx} argument to your drawing function has the standard JS type, {\tt
  CanvasRenderingContext2D}. That's not always true. Any drawing
done \emph{to the browser} is \emph{precisely} an object of this
type. If you only care about seeing your document within the browser
framework, and you don't care about generating figures for a
standalone {\tt .pdf} file, then that's all you need to know. Draw
with JS however you like. 

To produce TikZ output, the built-in {\tt CanvasRenderingContext2D}
must be ``spoofed.'' All of the drawing commands that normally go to
the built-in JS browser code go to \figput's {\tt CTX} class
instead, where they are converted to TikZ output.
This class implements a sub-set of the full list of drawing
commands normally available. The permitted commands are
\begin{verbatim}
  ctx.fill()
  ctx.stroke()
  ctx.lineWidth
\end{verbatim}
These work almost as they normally do. The {\tt lineWidth} is the
thickness of any paths, and can be thought of as being in \TeX's {\tt
  bp} units. The {\tt fill()} and {\tt stroke()} commands normally
take a JS {\tt Path2D} object, but you should pass an object of type
{\tt FPath} instead.

The methods of {\tt FPath} listed below are exactly like those found
in {\tt Path2D}. They add a new segment to an existing {\tt FPath},
obtained with {\tt new}. See any JS resource for an explanation. 
\begin{itemize}[itemsep=-1pt,leftmargin=*]
\item {\tt closePath()}
\item {\tt moveTo()}
\item {\tt lineTo()}
\item {\tt bezierCurveTo()}
\item {\tt ellipse()}
\end{itemize}
And these methods of {\tt FPath} are new -- they have no counterpart
in {\tt Path2D}.
\begin{itemize}[itemsep=-1pt,leftmargin=*]
\item {\tt addPath(p)} appends the {\tt FPath}, {\tt p}, to an
  existing {\tt FPath}. 
\item {\tt translate(p)} translates an {\tt FPath} by the given {\tt Point2D}.
\item {\tt rotate(angle)} rotates an {\tt FPath} by the given angle,
  in radians.
\item {\tt scale(s)} scales an {\tt FPath} by the given factor
  (relative to the origin).
\item {\tt reflectX()} reflects an {\tt FPath} across the $x$-axis.
\item {\tt reflectXY()} reflects an {\tt FPath} across the $x$ and $y$-axis.
\item {\tt rotateAbout(angle,p)} rotates a given {\tt FPath} about the
  given {\tt Point2D} by the given angle.

The methods above apply to an existing {\tt FPath} and return a new {\tt FPath}.
\item {\tt parametricToBezier(f,t0,t1,n)} converts a parametric
  function, {\tt f(t)}, to a path consisting of {\tt n} B\'ezier
  curves as   {\tt t} runs from {\tt t0} to {\tt t1}. The function
  must be defined to return {\tt Point2D} objects.

This is a static method, so it returns a new {\tt FPath} that is unrelated to any existing {\tt FPath} objects.
\end{itemize}
There are several additional methods in {\tt FPath} that may be
useful, but they haven't yet been tested well enough to mention
here (and they may change in the future). See the source code if
you're adventurous. 

Several of the methods above mention the {\tt Point2D} class, and it's what you would expect. Use
\begin{verbatim}
  new Point2D(x,y);
\end{verbatim}
to create these objects. There's a long list of methods defined in the
class: {\tt translate()}, {\tt rotate()}, {\tt length()},
\emph{etc}. See the source code for a complete list. 

Text is drawn in a way that is further from how it's normally done in
JS. Instead of calling {\tt ctx.fillText()}, there are three top-level
functions (not part of any class) for drawing text:
\begin{verbatim}
  drawText(ctx,txt,x,y,dx,dy);
  drawTextBrowserOnly(ctx,txt,x,y,dx,dy);
  drawTextTikZOnly(ctx,txt,x,y,dx,dy);
\end{verbatim}
The arguments are the same in the three cases, but the first function will draw
text to both the browser window and the TikZ, while the other two will
draw the text to only one destination or the other. The {\tt ctx} is
the same as the argument to your drawing function, {\tt txt} is the
string to be drawn, {\tt x,y} is the location of the text \emph{for
the browser}, relative to the lower-left corner of the
bounding-box. When the text is drawn to TikZ, it is drawn at {\tt
  x+dx,y+dy}. Because the fonts used in the browser and in TikZ differ
slightly, sometimes it's necessary to tweak the TikZ placement. Setting
\begin{verbatim}
  ctx.font = '10px san-serif';
\end{verbatim}
sets the browser font to something very close the default font used
by TikZ, but it's not a perfect match.  The {\tt dx} and {\tt dy}
arguments are optional and default to {\tt 0}.

As a bonus, the static method, {\tt Numerical.newton()}, can be used to
solve $f(x) = y$ for $x$, given $y$. It's a naive implementation of Newton-Raphson.

\subsection{JavaScript Widgets}

The foregoing framework could be used an an alternative to TikZ,
PSTricks, MetaPost, \emph{etc.}, if your aim is strictly static
output. To make the figures interactive, there must be some means to
accept user input -- widgets.

All widgets have a static {\tt register()} method. Do not use {\tt
  new} to create widgets! Registering a widget has the effect of
making it available for user input, and drawing it too (with one
exception). The arguments to these {\tt register()} methods vary with
the widget, but the first argument is always the {\tt ctx} known to
the drawing function, and the last argument is always a {\tt name} (a
string). One drawing could have several widgets of the same kind and
{\tt name} is used to distinguish 
them. So, if a particular drawing has three buttons, then they could
be created/registered with
\begin{verbatim}
let b1 = ButtonWidget.register(ctx,...,"first");
let b2 = ButtonWidget.register(ctx,...,"second");
let b3 = ButtonWidget.register(ctx,...,"third");
\end{verbatim}
These names only need to be unique within a single drawing and one
type of widget. So, in the example above, two {\tt
  DraggableDotWidget}s could be registered in the same drawing as
the {\tt ButtonWidget}s with
\begin{verbatim}
let dd1 = DraggableDotWidget.register(ctx,...,"first");
let dd2 = DraggableDotWidget.register(ctx,...,"second");
\end{verbatim}

\subsubsection{\tt ButtonWidget}

These are registered with 
\begin{verbatim}
ButtonWidget.register(ctx,x,y,text,name);
\end{verbatim}
The {\tt x,y} argument is the location of the button, relative to the
lower-left corner. The {\tt text} is what is shown as the message
within the button. Buttons work by calling back to your drawing code
whenever they are clicked. Every time the button is clicked, the
entire drawing function is executed.

There are two useful fields of {\tt ButtonWidget}. If {\tt b} is a
{\tt ButtonWidget}, then {\tt b.clickState} toggles between {\tt true} and {\tt
  false} with every click of the button, while {\tt b.resetState} is
set to {\tt true} whenever the button is clicked -- and remains {\tt
  true} until your drawing code resets it.

\subsubsection{\tt NumberInputWidget}

This is similar to {\tt ButtonWidget}, and is used to input single
numbers. It's registered with 
\begin{verbatim}
NumberInputWidget.register(ctx,x,y,v,name);
\end{verbatim}
The {\tt v} argument is the initial numerical value shown in the
widget. Whenever the user changes this value, the widget calls back
and executes the entire drawing function. To obtain this value, call
{\tt getValue()} on the relevant widget. Because this widget relies
on HTML, {\tt getValue()} returns a string. It may be necessary to
call the JS {\tt parseFloat()} or {\tt parseInt()} functions on this
value. 

\subsubsection{\tt DraggableDotWidget}

A ``draggable dot'' is a small dot that can be grabbed by the mouse
and moved around. They're registered with
\begin{verbatim}
DraggableDotWidget.register(ctx,x,y,name);
\end{verbatim}
To determine where the user has moved a dot, refer to the widget's
{\tt widgetX} and {\tt widgetY} fields. 

Most widgets are used ``off to the side'' and wouldn't normally
interfere with the way the figure is drawn, but these dots are typically
used as part of the drawing itself. The order in which dots are drawn
may matter. Sometimes a dot should be drawn first, so that other
elements of the figure may be drawn over it; and sometimes a dot
should be drawn last so that nothing can obscure it. For this
reason, unlike other widgets, registering the widget does not also
draw the widget. An explicit call to the widget's {\tt
  draw(ctx)} method must be made to draw the dot.

\subsubsection{\tt DraggableDrawWidget}

This is similar to {\tt DraggableDotWidget}, but it has no default
drawing behavior. The user must provide functions to specify what is
to be drawn. So it can also be a draggable polygon, or a draggable
anything-you-can-draw. Register these with
\begin{verbatim}
DraggableDrawWidget.register(ctx,x,y,
    drawFcn,drawSelFcn,testPosFcn,
    name);
\end{verbatim}
The {\tt drawFcn} must be defined to take the {\tt ctx} as the sole
argument. It can draw whatever it wants, noting that the origin is
shifted so that drawing should typically take place relative to {\tt
  (0,0)}, not {\tt (widgetX,widgetY)}. In most cases, this means
that the drawing code is independent of where the item is on the
page. {\tt drawFcn} must return an {\tt FPath} object to indicate
the ``clickable area'' for the item. For example, if what is being
drawn is a small square, then the function might be defined as
\begin{verbatim}
function drawSquare(ctx) {
  let p = new FPath();
  // p.lineTo, moveTo, etc., to make a square.
  ctx.fillPath(p);
  return p;
}
\end{verbatim}
The value returned is noted, and any click in the interior of the path
-- as determined by the JS function {\tt isPointInPath()} -- 
is taken to be a valid click on the object, to select it and drag it around.

The {\tt drawSelFcn} argument to {\tt register} is identical to the
{\tt drawFcn} argument, but it is called to draw the item when it is
``selected.'' Often, the only difference between the two functions
will be the color used to draw the item.

The {\tt testPosFcn} argument is used to determine whether a
particular location for the item is acceptable. For example, an item
could be limited to a particular region. This is called every time the
mouse is moved, after the item has been selected, up until the mouse
button is released. The function takes five arguments and returns a
boolean. The five arguments are the proposed {\tt x,y} position (the
mouse location), followed by the width of the drawing area, then the
height above and below the $x$-axis, where the width is equal to the
text width. Return {\tt true} if the 
proposed {\tt x,y} is acceptable; {\tt false} otherwise. At a minimum,
the function should typically check that {\tt x,y} is inside the
figure rectangle.

\subsubsection{\tt LoopAnimWidget}

This widget and the next one ({\tt OpenAnimWidget}) are for
animations, and work similarly. A {\tt LoopAnimWidget} is used for
animations that run in a repeating loop. There are many arguments to
register them.
\begin{verbatim}
LoopAnimWidget.register(ctx,x,y,scale,visWidget,
    steps,start,timeStep,
    visSteps,visFastSlow,visPauseRun,visCircle,triGrab,
    name);
\end{verbatim}
The {\tt x,y} values are the location of the center of the circle that
is the primary element of the widget. The widget can be made larger or
smaller by changing the {\tt scale} -- a value of {\tt 1.0} makes the
circle 40~{\tt bp} in radius. The {\tt visWidget} argument is a
boolean; set it to {\tt false} to run an animation without
showing the user this widget.

{\tt steps} is the integral number of steps that make up one loop of
the animation. {\tt start} is the step on which the animation
starts -- so that the animation starts with the {\tt start}-th
frame. {\tt timeStep} is the number of milliseconds between
frames. Unless the machine is \emph{very} fast, anything less than
{\tt 10} for {\tt timeStep} is probably pointless, and a setting of
{\tt 50} or {\tt 75} is more reasonable considering the human eye.

The next group of arguments -- {\tt visSteps} through {\tt triGrab} -- are
all booleans. They determine whether certain parts of the widget are
visible. {\tt visSteps} shows or hides an area above the circle that
can be used to change the number of steps taken per frame. The user
can use this control to skip frames of the animation. {\tt
  visFastSlow} is for some ``sideways chevrons,'' in an area below the circle,
that allow the user to change the number of milliseconds per
frame. {\tt vsPauseRun} is for a pause/run control below the
circle. {\tt visCircle} is for the entire circle. The circle has a
small grabable triangle to indicate which frame is being shown; use
{\tt triGrab} to show or hide the triangle.

To draw a particular frame from your drawing function, refer to the
{\tt curStep} field of the object. This will be a value in the range
from 0 to the {\tt steps} argument to {\tt register()}. 

\subsubsection{\tt OpenAnimWidget}

This widget is for open-ended animations that don't run in a loop,
but continue ``forever.'' Whereas {\tt LoopAnimWidget} indicates the
frame being viewed relative to a circle, {\tt OpenAnimWidget} uses a
bar, something like a scroll-bar. Registration is similar to
{\tt LoopAnimWidget}: 
\begin{verbatim}
OpenAnimWidget.register(ctx,x,y,scale,width,visWidget,
    timeStep,decay,
    visSteps,visFastSlow,visPauseRun,visBar,barGrab,
    name);
\end{verbatim}
The {\tt ctx}, {\tt x}, {\tt y}, {\tt scale} and {\tt visWidget}
arguments are as for {\tt LoopAnimWidget}. {\tt width} sets the width
of the bar, in {\tt bp}.

{\tt timeStep} is in milliseconds, as for {\tt LoopAnimWidget}. The
{\tt decay} is used to adjust the rate of travel of the ``thumb'' along the
bar as the animation progresses. Since the animation is potentially
infinite, the thumb moves quickly early in the animation, then moves
more and more slowly, so that it never quite reaches the end of the
bar. The position of the thumb is given by
$$1 - {1\over (1+{\tt decay})^s},$$
expressed as a fraction of the bar's total length, where $s$ is the
frame count (\emph{i.e.}, {\tt curStep}). A reasonable setting for {\tt
  decay} is something in the range of $10^3$ to $10^6$, although it
will depend on the animation. 

The arguments from {\tt visSteps} to {\tt barGrab} are all boolean and
control which controls that make up the widget are visible. They're
like their counterparts in {\tt LoopAnimWidget}, except that {\tt
  OpenAnimWidget} uses a bar instead of a circle.

\section{Distributing a Complete Document}
\label{sec-distributing}

The PDF document is no different than any other PDF, and can be
distributed in the usual ways. When moving the document in interactive
form to a public-facing website, a few steps are necessary.

See the {\tt javascript/release} directory. It has two files: {\tt
  figput.html} and {\tt figput.js}. The {\tt .js} file is just the
four files used for the local version ({\tt main.js}, \emph{etc}.),
consolidated to a single file and minified. These two files must
appear on the web-server in the same directory as the various files
that make up your document.

{\tt figput.html} is only a few lines long, and there are two
important changes that must be made. See the line that reads
\vskip 0.2cm
\noindent\verb!<body onload="doOpenDocument('unknown1',unknowny)" id="mainbody">!
\vskip 0.2cm
\noindent Change {\tt unknown1} to your document's name, \emph{without} the
{\tt .pdf} suffix. If the document was generated from {\tt
example.tex}, then {\tt unknown1} should be changed to {\tt example}.
Second, change {\tt unknowny} to {\tt 0}. When the document is
opened, this is the vertical position, in {\tt bp}, at which the document is
opened. Setting {\tt unknonwy} to {\tt 0} means that the document
will open at the top of the first page. Of course, it's possible
that you want the document to open at some other location, in which
case use some other value for {\tt unknowny}.

The server needs {\tt figput.html} (after the modifications
above), {\tt figput.js}, and the files that make up your
document. These files consist of the {\tt .pdf} file, the {\tt .fig.aux}
file, any {\tt .js} files referred to by \verb=\LoadFigureCode=, plus
any {\tt .fjs} files generated by your document. The server does not
need the {\tt .tikz} files. 

\section{Code Internals}

An understanding of this section isn't necessary to use \figput; it's
here to provide a bit of background for anyone who wants to improve
or modify the framework. Most of \figput\ is in TypeScript, which
should make it easier to follow.

Throughout the various files that make up the system are
{\tt BUG} annotations. These are not bugs \emph{per se}, but are more
like infelicities, opportunities for improvement, or to draw attention
to something confusing or problematic. If you want to contribute to
\figput, then these are a good place to start.

\subsection{\tt figput.sty}

The {\tt .sty} file is well-commented and should be
self-explanatory. However, I am far from an expert in \LaTeX3
programming. Here is a high-level list of ways in which it might be
possible to improve things.

\begin{itemize}[itemsep=-1pt,leftmargin=*]
\item {\tt figput.sty} is written for \LaTeX3. The entire thing might
  be simpler if it were in Lua\TeX.
\item The ability to explicitly specify the inner and outer margins
  used by \figput\ (with \verb=\SetInnerMargin= and
  \verb=SetOuterMargin=) is useful, but it would often suffice to
  default to the margins used by \LaTeX. How can those values be
  obtained?
\item Someone who is more expert in \LaTeX3 could improve the code's
  clarity and brevity.
\item It would be nice if \figput\ could create figures within a {\tt
  minipage}, but that would require extensive changes to both the
  \LaTeX\ and browser code.
\end{itemize}

\subsubsection*{A Quirk}

\LaTeX\ on Windows has a quirk. \TeX\ will not write to a
file whose type is executable. Thus, it won't write to
a {\tt .pl} file (Perl), {\tt .py} (Python) file, \emph{etc.},
depending on how the machine is set up. \figput\ creates
JavaScript files, which are typically saved as {\tt .js} files, and
these may be seen as executable by Windows. Fortunately,
browsers doesn't care what the suffix is, so \figput\ uses
{\tt .jfs} as the suffix.  
 
This limitation on the file name suffix applies to \verb=\write=, but
not to \verb=\write18=. \TeX\ is limited in this way to prevent
accidental errors and for security. One solution might be to have the
package write files using an acceptable suffix, then change the
suffix with \verb=\write18=, which does not have this
limitation. However, to use \verb=\write18= this way, \LaTeX\ would
have to be invoked with something like
\begin{verbatim}
pdflatex -shell-escape
\end{verbatim}
which would be annoying. 

\subsection{\tt server.py}

The server is a simple thing, so there's not much to say. It
would be nice if it were written in a compiled language so
that the user doesn't need to install Python.

\subsection{Browser Code}

This part is written in TypeScript (TS), which is essentially JS with
type annotations. It must be translated 
to JS for the browser to run it, but TS code is easier to follow, and
the type information heads off many programming errors.

Although the program is (to my knowledge!) bug-free in the usual
sense, there are plenty of {\tt BUG} annotations to draw attention to
places where the code could be improved. Some of the higher-level
possibilities for improvement are noted below.
\begin{itemize}[itemsep=-1pt,leftmargin=*]
\item \figput\ relies on the {\tt pdf.js} library for rendering
  PDFs. The result is not as crisp as I would like. It's unclear
  whether the problem is with the library itself, or how it is being
  used. Someone who knows that library well may be able to improve the
  output. Maybe a different library would work better.
\item Most of the widgets do not rely on the DOM, but {\tt
  NumberInputWidget} and {\tt ButtonWidget} are based on the built-in
  DOM elements. It would be nice if these didn't use the DOM.
\item Many additional widgets could be created.
\item Adjust the font used for figures in the browser to make it more
  nearly identical to what appears in the PDF. 
\item Color is entirely neglected on the \LaTeX\ end of \figput. Browser output
  can be colored arbitrarily (with {\tt ctx.fillStyle} or {\tt
    ctx.strokeStyle}, as usual), but the TikZ output is limited to
  black and white. 
\item Double-buffer everything. Animations seem to run fine,
  without any noticeable flicker, but double-buffering \emph{all}
  drawing would put an end to any concerns about flicker.
\item Animations should be turned off and stop generating events when
  they are scrolled out of view. This would save CPU cycles, which may
  matter for documents with many animations.
\end{itemize}

\noindent And some items that are more ambitious...
\begin{itemize}[itemsep=-1pt,leftmargin=*]
\item Allow figures to be set within a {\tt minipage} (mentioned above).
\item Double-clicking a figure (or something) could enlarge the figure
  to give more room for interaction.
\item Come up with a scheme so that the code that draws a figure could
  be made part of the document. For certain topics, the code that
  draws a figure can be as instructive as the figure itself.
\item Extend the framework somehow so that it can be used within an
  IDE. Code that's mathematically-oriented is difficult to
  document within source code using simple text. It would be easier to
  understand if it's explained as \LaTeX\ output that appears 
  in the source code, rather than in a separate document.
\item Extend the framework to allow inclusion of snippets of
  \LaTeX\ output in standard HTML documents. Things like KaTeX and
  MathJax are convenient, but incomplete relative to \LaTeX.
\end{itemize}

\newpage

\subsection*{Version History}

\begin{itemize}[itemsep = -3pt]
\item{0.90} -- July 21, 2022. I've been using this, privately, for
  several months. Numerous improvements and extensions are possible,
  but it definitely works.
\end{itemize}


\vskip 0.3cm
\hrule
\vskip 0.3cm


Thanks to Dan Pratt for insights regarding TypeScript, and
improvements to the code.

Several people on {\tt tex.stackexchange} very
patiently answered numerous questions. Thank you.



\end{document}
