
\documentclass[10pt]{article}

% This file was written to demonstrate this package.
\usepackage{figput}

% Allows various tweaks to itemize as optional arguments.
\usepackage{enumitem}

% Allows \verb in footnotes
\usepackage{fancyvrb}


\begin{document}

% Need this for fancyvrb to work.
\VerbatimFootnotes


% Getting this right may involve some experimentation.
\SetInnerMargin{135pt}
\SetOuterMargin{135pt}

% Turn this on to show figures or off to save time.
\NeverSkip

% Some of the \FigPut commands refer to this code.
\LoadFigureCode{externalcode.js}


This document demonstrates some of the features of \textsc{FigPut}. It
consists of a series of snippets of exposition. There are a few simple
examples -- B\'ezier curves, the ellipse and diffiusion -- followed by
a more extended example about gears.

% Much of what follows uses the figput *environment*. This uses the
% \FigPut *command*. 
\begin{figure}[b!]
\FigPut{bezier,175bp}[17bp,22bp,done,skip]
\caption{A Cubic B\'ezier Curve}
\label{fig-cubic-bezier}
\end{figure}

\section{Example: B\'ezier Curves}

B\'ezier curves are a convenient way of defining curves in the
plane. They can be defined using Bernstein polynomials. The 
Bernstein polynomials of degree $n$ are defined by
$$B_i^n(t) = {n \choose i} t^i (1-t)^{n-i},\qquad i=0,\ldots,n.$$
A cubic B\'ezier curve is a linear combination of the Bernstein
polynomials of degree 3:
$$C(t) = \sum_{i=0}^3 p_i B_i^3(t),$$
where the four $p_i$ are points in the plane and $t$ is limited to
$[0,1]$. This can be written as the pair of equations $(x(t),y(t))=
C(t)$, where
\begin{eqnarray*}
x(t) &=& x_1\cdot (1-t)^3 + x_2\cdot 3(1-t)^2t + x_3\cdot 3(1-t)t^2 +
x_4\cdot t^3 \\
y(t) &=& y_1\cdot (1-t)^3 + y_2\cdot 3(1-t)^2t + y_3\cdot 3(1-t)t^2 +
y_4\cdot t^3,
\end{eqnarray*}
$(x_i,y_i) = p_i$ and the indices on the $p_i$ have been
shifted. We have $C(0) = p_1$ and $C(1)=p_4$ and the other two points
act as ``controls.'' See Figure (\ref{fig-cubic-bezier}), noting that
the line determined by the pair of points controlling each end of the curve is
tangent to the corresponding end-point of the B\'ezier curve.

\section{Example: Drawing an Ellipse}

One way to define an ellipse is illustrated by Figure
(\ref{fig-ellipse-draw}). Choose two points, $F_1$ and $F_2$ (the
{\bf foci}), and fix some $k>0$. The ellipse is then
the set of points, $P$, satisfying
$$d(P,F_1) + d(P,F_2) = k,$$
where $d(A,B)$ is the distance from $A$ to $B$.

The figure makes it clear why this is sometimes refered to as the
``tacks and string'' definition. Imagine tacking each end of a bit of
string, of length $k$, to the two foci; then tracing out the ellipse
by holding a pencil at the limit of what the string will allow as the
pencil travels about the foci. 

\begin{figure}
\begin{figput}{elldraw,175bp}[done,skip]
  function elldraw(ctx) {

  // The length of the string.
  let k = 150;
  
  let yaxis = 85;
  
  // Limit these two points to be along the line y = yaxis.
  let w1 = DraggableDotWidget.register(ctx,125,yaxis,'d1');
  let w2 = DraggableDotWidget.register(ctx,225,yaxis,'d2');

  // We need a LoopAnimWidget for an animation, even though we never want
  // see the widget.
  let numSteps = 1000;
  let loopw = LoopAnimWidget.register(ctx,-100,100,1.0,
  false, // hidden
  numSteps, // steps per loop
  0, // starting step
  10, // ms time step
  false,false,false,false,false, // nothing is visible
  'loop');

  w1.widgetY = yaxis;
  w2.widgetY = yaxis;
  
  // The origin is the mid-point.
  let cx = (w1.widgetX + w2.widgetX) / 2;
  let cy = yaxis;

  // c = distance from foci to center, a = x-radius and b = y-radius.
  let c = Math.abs(w1.widgetX - cx);
  let a = k / 2;
  let b = Math.sqrt(a*a - c*c);

  // The ellipse
  let p = new FPath();
  p.ellipse(cx,cy,a,b,0,0,2*Math.PI);
  ctx.strokeStyle = 'black';
  ctx.lineWidth = 1.5;
  ctx.stroke(p);

  // An animation, based on the parameterization p(t) = (a cos(t),b sin(t)).
  // This based on the eccentric angle, so time won't seem to be
  // constant, but that's OK. This is easier.
  let t = 2*Math.PI * loopw.curStep / numSteps;
  let px = a * Math.cos(t) + cx;
  let py = b * Math.sin(t) + cy;

  p = new FPath();
  p.moveTo(cx - c,yaxis);
  p.lineTo(px,py);
  p.lineTo(cx + c,yaxis);
  ctx.lineWidth = 0.4;
  ctx.stroke(p);
  
  // In this case (unlike most others), the widgets must be explicitly drawn.
  w1.draw(ctx);
  w2.draw(ctx);
}    
\end{figput}
\caption{Tacks-and-string Ellipse}
\label{fig-ellipse-draw}
\end{figure}

\section{Example: Diffusion}

The concept of diffusion is illustrated by Figure
(\ref{fig-boltzmann}). Statistical mechanics and Boltzmann's equation
explain concepts like heat transfer and the gas laws by modeling the
random motion of many particles. These equations may be difficult to grasp,
but an intuitive understanding is not difficult. At time zero, there
is some set of particles on the left half of the box, each of which is
moving with some randomly distributed momentum. The divider is removed
and, over time, the particles distribute themselves more evenly
throughout the box.  

\begin{figure}
\begin{figput}{diffusion,180bp}[0bp,30bp,done,skip]
function diffusion(ctx) {
  
  // This is an open-ended animation.
  let wanim = OpenAnimWidget.register(ctx,50,-32,1.0,
    200, // horizontal size
    true, // visible
    10, // time step (ms)
    10000, // decay
    false, // steps not visible
    true, // fast/slow visible
    true, // pause/run visible
    true, // time bar visible
    false, // time bar not grabable.
    'anim');

  // A button for removing the divider.
  let wbut = ButtonWidget.register(ctx,-85,32,'Open/Restart','but');

  // Draw the enclosing box.
  let p = new FPath();
  p.moveTo(1,2);
  p.lineTo(300,2);
  p.lineTo(300,178);
  p.lineTo(1,178);
  p.closePath();
  ctx.lineWidth = 3;
  ctx.fillStyle = 'black';
  ctx.strokeStyle = 'black';
  ctx.stroke(p);

  // These define the box for the particles.
  let boxleft = 5;
  let boxright = 150;
  let boxbot = 4;
  let boxtop = 175;
  let boxdivider = 145;

  if ((diffusion.started === undefined) || (wbut.resetState === true))
    {
      wbut.resetState = false;
      wanim.curStep = 0;
      
      // Allocate 100 dots in the left half of the box.
      // Each needs a position and direction.
      diffusion.pos = [];
      diffusion.dir = [];
      for (let i = 0; i < 100; i++)
        {
          // Position
          let r = Math.random();
          let x = boxleft + r*(boxdivider - boxleft - 4);

          r = Math.random();
          let y = boxbot + r*(boxtop-boxbot)
          diffusion.pos.push({x: x, y: y});

          // Direction vector. Say each is evenly distributed over
          // (0,1). That is not physically correct, but that's OK.
          // To make this better, consider how the balls could bounce
          // off of each other. That's messy, so I let them pass
          // through one another, and only consider bouncing off walls.
          let vx = (Math.random() - 0.50) / 2.0 ;
          let vy = Math.sqrt(0.25*0.25 - vx*vx);
          if (Math.random() > 0.50)
            vy *= -1;

          // Don't let either value be too small. It looks weird.
          if (Math.abs(vx) < 0.05)
            {
              if (vx < 0.0)
                vx = -0.05;
              else
                vx = 0.05;
            }
          if (Math.abs(vy) < 0.05)
            {
              if (vy < 0.0)
                vy = -0.05;
              else
                vy = 0.05;
            }

          diffusion.dir.push({x: vx, y: vy});
        }

      diffusion.started = true;
    }

    // As noted, we only consider bouncing off walls. Assume that a ball
    // moves from it's old position to new by the dir vector with each
    // time step. If that would take the ball outside the rectangle, then
    // we must consider reflection.
    let dotR = 3;
    
    // Draw the "wall" if it's closed.
    if (wbut.clickState === false)
      {
        p = new FPath();
        p.moveTo(boxdivider,2);
	p.lineTo(boxdivider,boxtop + 2);
	ctx.lineWidth = 3;
	ctx.stroke(p);
      }
    
    // This depends on whether the wall is up.
    let rightEdge = boxdivider - 3;

    if (wbut.clickState === true)
      rightEdge = 299;

    for (let i = 0; i < diffusion.pos.length; i++)
      {
        let pold = diffusion.pos[i];
        let vold = diffusion.dir[i];

        let pnew = {x: pold.x + vold.x, y: pold.y + vold.y};
        //console.log(i+ ' ' +pnew.y);
        if (pnew.y < boxbot)
          {
            // Bounced off top edge.
            // If each step takes 1 unit of time, then it hit the edget at
            // Actually... be sloppy. I'm not really doing physics; I just
            // want to demonstrate the FigPut system.
            // The balls are moving slowly, so only allow them to reflect
            // at the time of the step divisions.
            diffusion.dir[i].y *= -1;
          }
          if (pnew.y > boxtop)
            {
              // Bottom edge bounce.
              diffusion.dir[i].y *= -1;
            }
          if (pnew.x < boxleft)
            // Left edge bounce.
            diffusion.dir[i].x *= -1;
          if (pnew.x > rightEdge - 2)
            // Right edge bounce.
            diffusion.dir[i].x *= -1;

        // Redo, after taking reflections into account.
        vold = diffusion.dir[i];
        pnew = {x: pold.x + vold.x, y: pold.y + vold.y};

        diffusion.pos[i] = pnew;

        p = new FPath();
        p.ellipse(pnew.x,pnew.y,dotR,dotR,0,0,2*Math.PI);
        ctx.fill(p);
      }
}    
\end{figput}
\caption{Diffusion and Boltzmann's Equation}
\label{fig-boltzmann}
\end{figure}

\newpage

\section{Example: Gears}

Gears like those in Figure (\ref{fig-gear-toy}) might work after a
fashion, but it's rare to see such gears in anything other than a
child's toy. As a rule, gears take the form shown in Figure 
(\ref{fig-final-gear}). Why? 

\begin{figure}[b!]
\begin{figput}{badgear,130bp}[done,skip]
function badgear(ctx) {
  
  var path1 = new FPath();
  
  let numTeeth = 17;
  let cx = 70;
  let cy = 65;
  let inR = 40;
  let outR = 50;
  
  path1 = new FPath();
  
  // A dot at the center.
  path1.ellipse(cx,cy,2,2,0,0,2*Math.PI);
  ctx.fillStyle = 'black';
  ctx.fill(path1);
  
  // The teeth of the gear.
  path1 = new FPath();
  
  // One degree, in radians, then 360 and 180 degrees.
  let d = Math.PI / 180;
  let d360 = 2 * Math.PI;
  let d180 = Math.PI;
  
  let x = cx + inR * Math.cos(0);
  let y = cy + inR * Math.sin(0);
  path1.moveTo(x,y);
  
  let deltaA = Math.PI/numTeeth;
  
  for (let i = 0; i < numTeeth; i++)
    {
      x = cx + inR * Math.cos(2 * i * deltaA + deltaA);
      y = cy + inR * Math.sin(2 * i * deltaA + deltaA);
      path1.lineTo(x,y);

      x = cx + outR * Math.cos(2 * i * deltaA + deltaA);
      y = cy + outR * Math.sin(2 * i * deltaA + deltaA);
      path1.lineTo(x,y);

      x = cx + outR * Math.cos(2 * i * deltaA + deltaA + deltaA);
      y = cy + outR * Math.sin(2 * i * deltaA + deltaA + deltaA);
      path1.lineTo(x,y);

      x = cx + inR * Math.cos(2 * i * deltaA + deltaA + deltaA);
      y = cy + inR * Math.sin(2 * i * deltaA + deltaA + deltaA);
      path1.lineTo(x,y);
    }
  
  path1.closePath();
  ctx.lineWidth = 1.5;
  ctx.stroke(path1);
  
  // Now the triangular teeth.
  numTeeth = 21;
  cx = 230;
  outR = 60;
  
  path1 = new FPath();
  path1.ellipse(cx,cy,2,2,0,0,2*Math.PI);
  ctx.fill(path1);
  
  path1 = new FPath();
  
  x = cx + + inR * Math.cos(0);
  y = cy + inR * Math.sin(0);
  path1.moveTo(x,y);
  
  for (let i = 0; i < numTeeth; i++)
    {
      x = cx + outR * Math.cos(i * d360/numTeeth + d180/numTeeth);
      y = cy + outR * Math.sin(i * d360/numTeeth + d180/numTeeth);
      path1.lineTo(x,y);
            
      x = cx + inR * Math.cos((i+1) * d360/numTeeth);
      y = cy + inR * Math.sin((i+1) * d360/numTeeth);
      path1.lineTo(x,y);
    }
  path1.closePath();
  ctx.stroke(path1);
}

\end{figput}
\caption{Crude Gear Designs}
\label{fig-gear-toy}
\end{figure}

Figure (\ref{fig-gear-spoke}) illustrates the fundamental problem of
gear design, using a particularly bad design. The ``gears'' here have
been reduced to something more like spokes. The gear on the right
rotates counter-clockwise at a constant rate and drives the gear on
the left. As the gears rotate, the rate of rotation of the gear on the
left will not be constant; at some positions, the left gear is nearly
stationary and at other positions, it rotates much faster than the
gear on the right. In addition, the point of contact slides,
generating friction and wear. 

\begin{figure}
\begin{figput}{badspokegear,130bp}[done,skip]
function badspokegear(ctx) {
    
  // Two gear-like pinwheels, where the one on the right drives the one on 
  // the left. The idea is to show how the rotation surges, and why something
  // like involvolute gears are needed.
  //
  // Modeling this is surprisingly fiddly. There's no need to make this
  // fully general, for all possible arrangements, and it's not too bad
  // if we restrict things a bit.
  // 
  // In general (too general), suppose that the two sets of spokes have
  // lengths R1 and R2. Think of the gear on the right driving the
  // gear on the left, with the gear on the right rotating ccw (under the
  // usual RH coordinate system). WLOG, assume that the two axles are on the
  // x-axis. We want to know where the two spokes initially touch "at
  // the top," but it's actually easier to figure out where the last touch
  // "at the bottom."
  //
  // To simplify, assume that all spokes (on both gears) have the same
  // length, R. And assume that the centers of these gears are at
  // (-x,0) and (x,0) so that the centers are 2x appart. Obviously, R > x
  // for the gears to mesh at all.
  // 
  // Since the spokes are all the same length, it's easy to see that
  // The angle of the RH spoke at the top is alpha above the x-axis when
  // the spokes initally touch, and alpha below the x-axs when they cease
  // to touch. Here, alpha is defined by
  // cos(alpha) = x/R.
  //
  // Furthermore, let delta be the angle from spoke to spoke, so the
  // number of spokes is 2 pi / delta. Assume that both gears have the
  // same number of spokes. Note that delta must be less than 2 alpha
  // for the spokes to always be in contact. Ideally, I want delta to be
  // large enough so that no more than one pair of spokes (one on left and
  // on on right) is ever in contact. Otherwise, it's more messy since
  // you'd have to figure out which pair of spokes is doing the driving.
  // In fact, it looks like I want delta = 2 alpha, or to put it another
  // way, given delta, it follows that alpha = delta / 2.
  //
  // Now, think about what happens as the point of contact goes from the
  // top, where the spoke barely touch, downward. The point of contact
  // stays on the top of the left spoke and slides down on the right spoke.
  // What you have is a triangle on the left with hypoteneuse R, and a 
  // triangle on the right with hypoteneuse r. We need to determine r.
  // Let beta be the angle of the right spoke above the x-axis. beta
  // grows linearly with time, and is fully known...actually, beta
  // *shrinks* if we think of it as the angle of the right spoke above
  // the x-axis.  
  //
  // Let c be the x-coordinate of the point of contact, and let x1 be
  // the distance from c to -x (the left axle) and x2 be the distance
  // from c to +x (the right axle). We must have
  // R^2 - x1^2 = r^2 - x2^2
  // since the left and right triangles share the same vertical leg.
  // And x1 + x2 = 2x by construction. The third equation we need
  // is cos(beta) = x2/r or x2 = r cos(beta). 
  // 
  // Solve these three equations for x1, x2 and r in terms of R and x.
  // First, use x1 = 2x - x2. Substitute:
  // R^2 - x1^2 = r^2 - x2^2
  // R^2 - (2x-x2)^2 = r^2 - x2^2
  // R^2 - 4x^2 + 4x*x2 - (x2)^2 = r^2 - x2^2
  // R^2 - 4x^2 + 4x*x2 = r^2
  //
  // Now use x2 = r cos(beta) and use C = cos(beta) for brevity:
  // R^2 - 4x^2 + 4x*x2 = r^2
  // R^2 - 4x^2 + 4x*(rC) = r^2
  // r^2 - 4Cx r + 4x^2 - R^2 = 0.
  // Solve for r by quadatic equation:
  // r = (1/2) [ 4Cx \pm \sqrt{ 16C^2x^2 - 4(4x^2-R^2) } ]
  // r = 2Cx \pm \sqrt{ 4C^2x^2 - 4x^2+ R^2 }
  // r = 2Cx \pm \sqrt{ 4x^2(cos^2(beta) - 1 ) + R^2 }
  // r = 2Cx \pm \sqrt{ -4x^2 sin^2(beta) + R^2 }
  //
  // Aside: note that the discriminant is always postive since R > x.
  // I'm not sure what the two solutions here might "mean."
  //
  // This tells us what is going on from the point when the spokes touch
  // at the top until you reach the point where they are aligned along the
  // x-axis. However, we need the angle on the left -- that was the 
  // whole point. We have angle beta on the right, and we want to know
  // the angle, gamma, on the left.
  //
  // We have cos(gamma) = x1/R, where gamma is measured upwards from the 
  // x-axis to the spoke in contact on the left gear (like normal).
  //
  // Now do something similar as the point of contact goes below the
  // x-axis. It's the same basic calculation, except that the tip of the
  // right spoke is in contact somewhere along the left spoke.
  // Left and right are reversed.
  //
  // We still have x1 + x2 = 2x but now R^2 - x2^2 = r^2 - x1^2, and
  // cos(beta) = x2/R. It's the same basic game. We get.
  // 
  // R^2 + 4x^2 - 4x*x2 = r^2
  // just as before. But now use x2 = R cos(beta) (R, not r) and it's
  // a simpler problem:
  // R^2 - 4x^2 + 4x*x2 = r^2
  // R^2 - 4x^2 + 4x R cos(beta) = r^2
  // becomes
  // r = \pm\sqrt{ R^2 - 4x^2 + 4x R cos(beta) }
  //
  // BUG: The calculation above treats the spokes as geometric lines, with zero
  // thickness. Ideally, I should take the width of the lines into account.
  
  // The presence of this widget means that this is an animation.
  // The code here will be called on a regular schedule.
  let stepsPerRev = 500;
  let startStep = 0;
  let timeStep = 20;
  
  let w = LoopAnimWidget.register(ctx,
    -LoopAnimWidget.Radius,LoopAnimWidget.Radius + LoopAnimWidget.TopHeight,0.75,
    true,stepsPerRev,startStep,timeStep,
    true,true,true,true,true,"wigname");

  // BUG: I messed up when I wrote this and inadvertantly did all the
  // measurements to the wrong scale. Thus, I have to undo my mistake.
  // The real fix is to change all the constants below so that this
  // scale() isn't necessary.
  let mistakeT = ctx.getTransform();
  ctx.scale(0.5,0.5);
  
  // The two gears are on opposite sides of this vertical line.
  let xZero = 300;
  
  // You can change the number of teeth, but the surging is most apparent
  // with four teeth (which is the minimum geometrically possible given
  // the algebra. Somewhere around 6 or 7 teeth it becomes hard to see. 
  // BUG: five teeth generates some kind of "jump" in the animation. 
  // Not sure why.
  let numTeeth = 4;
  let spokeLength = 100;
  
  // The angle between spokes.
  let delta = 2*Math.PI/numTeeth;
  
  // Spoke radius and thickness.
  let R = spokeLength;
  let spokeWidth = 2;
  
  // Distance of each spoke axis from the center line.
  // I want the top pair of spokes to make contact at the same instant 
  // that the bottom pair of spokes ceases contact. That requires
  let x = R * Math.cos(delta/2);
  
  // Draw the spokes on the right. These are the drivers, so the angle rises
  // linearly with time.
  let cx = (xZero + x);
  let cy = 140;
  
  var path1 = new FPath();
  
  // A dot at the center, probably not necessary.
  path1.ellipse(cx,cy,2.0*spokeWidth,2.0*spokeWidth,0,0,2*Math.PI);
  ctx.fill(path1);
  
  // Angle by which the gear has rotated, based on the step of the animation.
  let rotA = w.curStep * 2 * Math.PI / stepsPerRev;
  
  for (let i = 0; i < numTeeth; i++)
    { 
      path1 = new FPath();
      
      let angle = i*delta + rotA;
      while (angle < 0)
        angle += 2 * Math.PI;
      while (angle > 2*Math.PI)
        angle -= 2 * Math.PI;
      
      path1.moveTo(cx,cy);
      let x = cx + R * Math.cos(angle);
      let y = cy + R * Math.sin(angle);
      path1.lineTo(x,y);
      
      ctx.strokeStyle = "black";
      ctx.lineWidth = spokeWidth;
      ctx.stroke(path1);
    }
  
  // Now the left, driven, set of spokes.
  cx = (xZero - x);
  
  // Figure out which of the spokes on the right gear is the one in contact
  // and doing the driving. It's the one whose angle relative to the 
  // x-axis is between delta/2 above and delta/2 below the axis.
  // To put it another way, its angle is in the range pi\pm delta/2.
  // This could have been noted in the loop above.
  let contactAngle = 0.0;
  for (let i = 0; i < numTeeth; i++)
    {
      let angle = i*delta + rotA;
      while (angle < 0)
        angle += 2 * Math.PI;
      while (angle > 2*Math.PI)
        angle -= 2 * Math.PI;
           
      if ((angle < Math.PI + delta/2) && (angle > Math.PI - delta/2))
        {
          contactAngle = angle;
          break;
        }
    }
  
  // contactAngle is the angle of the right gear's "contact spoke," but
  // measured from the right. I want beta (in the notation above), measured
  // from the left side, to put it in the range [-delta/2,+delta/2].
  contactAngle = Math.PI - contactAngle;
  
  var gamma = 0;
  if (contactAngle >= 0)
    {
      // The point of contact is above the x-axis. Slightly messier case.
      let beta = contactAngle;
      let disc = R*R - 4*x*x*Math.sin(beta)*Math.sin(beta);
      let root = Math.sqrt(disc);
      
      // Conceptually, I'm not sure why, but this seems to be the right
      // choice -- minus, not plus.
      let r = 2 * Math.cos(beta) * x - root;
      
      let x2 = r * Math.cos(beta);
      let x1 = 2 * x - x2;
      
      // I didn't address this in the long comment above, but
      // (x1,sqrt(R^2-x1^2)) is the end-point of the left spoke in
      // contact with the right spoke. The angle for this spoke is thus
      // gamma, where cos(gamma) = x1/R.
      gamma = Math.acos(x1/R);
    }
  else
    {
      // The point of contact is below the x-axis, hence the minus to 
      // make beta positive.
      let beta = -contactAngle;
      let disc = R*R + 4*x*x - 4*x*R*Math.cos(beta);
      let r = Math.sqrt(disc);
      
      let x2 = R * Math.cos(beta);
      let x1 = 2 * x - x2;
      
      gamma = 2*Math.PI - Math.acos(x1/r);
    }
  
  // A dot at the center, probably not necessary.
  path1 = new FPath();
  path1.ellipse(cx,cy,2.0*spokeWidth,2.0*spokeWidth,0,0,2*Math.PI);
  ctx.fill(path1);
  
  // Draw the left spokes, based on gamma.
  rotA = -w.curStep * 2 * Math.PI / stepsPerRev;
  
  for (let i = 0; i < numTeeth; i++)
    { 
      path1 = new FPath();
      
      path1.moveTo(cx,cy);
      let x = cx + R * Math.cos(i*delta + gamma);
      let y = cy + R * Math.sin(i*delta + gamma);

      path1.lineTo(x,y);

      ctx.lineWidth = spokeWidth;
      ctx.stroke(path1);
    }
  
  // BUG: Undo the above fix of my scaling mistake.
  ctx.setTransform(mistakeT);
}    
\end{figput}
\caption{Maybe the Worst Possible Gears?}
\label{fig-gear-spoke}
\end{figure}

Understanding how to design gears without the kind of surging motion
and wear inherent in the gears of Figure (\ref{fig-gear-spoke})
explains why nearly all modern gears take the form of Figure
(\ref{fig-final-gear}). The most glaring problem with the
spoke-like gears is the way their motion varies -- imagine riding in a
car with a gear-train based on such gears!

Gears typically have what's called \emph{conjugate action},
meaning that the ratio of their rates of rotation is constant,
with no surging or lagging. Sometimes this is also called the
\emph{fundamental law of gearing}, although it would be more accurate
to call it a ``commonly desired feature,'' rather than a ``law.''
Arranging the teeth of gears so that they have conjugate action is
surprisingly tricky.

\subsection{The Involute}
  
Figure (\ref{fig-basic-gear-constraints}) shows a cam and a lever-arm
pushing against each other, causing them to rotate about their
respective axes. Imagine that the cam rotates counter-clockwise,
pushing the arm downwards. There are several crucial observations:
\begin{enumerate}[itemsep=-1pt]
\item The two curves must be tangent at the point of contact.
\item The force from one part to the other must be directed along a
  line perpendicular to the two curves at the point of contact. Call
  this the \emph{line of action}.
\item Let $P$ be point where the line of action intersects the line
  connecting the two centers of rotation. This is called the
  \emph{pitch point}. The instantaneous ratio of
  the two rates of rotation is equal to the ratio of the distances
  from $P$ to each of the centers of rotation.
\end{enumerate}
In conclusion, if two gears are to have conjugate action, then the
pitch point must be fixed.

\begin{figure}
\begin{figput}{gearconstraints,120bp}[done,skip]
function gearconstraints(ctx) {

  // A static figure. No widgets.
  //
  // I wrote this before I changed the framework to assume a RH coordinates
  // system. Instead of messing around changing the values below, each path
  // is reflected and translated, much like a transformation matrix.

  // Cam center.
  let cax = 250;
  let cay = 80;
  let ra = 5;

  let p = new FPath();
  p.ellipse(cax,cay,ra,ra,0,0,2*Math.PI);
  ctx.strokeStyle = 'black';
  ctx.lineWidth = 0.4;

  p = p.reflectX();
  p = p.translate(new Point2D(0,120));
  
  ctx.stroke(p);

  // Cam itself

  // Right upper
  let cs = 15; // space to right of axis
  let h1 = 60; // height above 
  let tan1 = 25; // tightness of tangents
  let tan2 = 25;

  p = new FPath();
  p.moveTo(cax+cs,cay);
  p.bezierCurveTo(cax+cs,cay - tan1,
    cax-cs + tan2,cay-h1,
    cax-cs,cay-h1);

  // left upper
  let h2 = 70; // how far left
  let t1 = 20; // thickness of arm
  tan1 = 4;
  tan2 = 3;
  p.bezierCurveTo(cax-cs-tan1,cay - h1,
    cax-cs - h2,cay - h1,
    cax-cs - h2,cay - h1 + t1);

  // Lower left
  let h3 = 40; // how far back right
  p.bezierCurveTo(cax-cs-h2,cay - h1 + t1 + tan2,
    cax-cs - h2,cay - h1 + 2*t1,
    cax-cs - h2 + h3,cay - h1 + 2*t1);

  // Inner concave
  tan1 = 20;
  tan2 = 20;
  p.bezierCurveTo(cax-cs-h2 + h3 + tan1,cay - h1 + 2*t1,
    cax-cs,cay - tan2,
    cax-cs,cay);

  // lower left below
  tan1 = 8;
  tan2 = 8;
  p.bezierCurveTo(cax-cs,cay + tan1,
    cax-tan2,cay + cs,
    cax,cay+cs);

  // lower right below
  // BUG: It's hard to get these curves to meet smoothly. Need a better 'close'.
  tan1 = 17;
  tan2 = 10;
  p.bezierCurveTo(cax + tan1,cay + cs,
    cax + cs,cay - tan2,
    cax + cs,cay);

  ctx.lineWidth = 2.0;

  p = p.reflectX();
  p = p.translate(new Point2D(0,120));
  
  ctx.stroke(p);

  // Now something similar for the long arm.
  // The easiest way to do this is to draw it horizontal, but rotate all the
  // points by some angle so that the arm touches the cam.
  cax = 30
  cay = 100;
  
  p = new FPath();
  p.ellipse(cax,cay,ra,ra,0,0,2*Math.PI);
  ctx.strokeStyle = 'black';
  ctx.lineWidth = 0.4;

  p = p.reflectX();
  p = p.translate(new Point2D(0,120));

  ctx.stroke(p);

  // top right
  cs = 12; // half thickness
  let L = 150; // length

  p = new FPath();
  tan1 = 10;
  tan2 = 10;
  p.moveTo(cax,cay + cs);
  p.bezierCurveTo(cax + tan1,cay + cs,
    cax + L - tan2,cay +cs,
    cax + L,cay + cs);

  // lower right
  tan1 = 6;
  tan2 = 6;
  p.bezierCurveTo(cax + L + tan1,cay +cs,
    cax + L + cs,cay + tan2,
    cax + L + cs,cay);

  // upper right
  p.bezierCurveTo(cax + L + cs,cay - tan1,
    cax + L + tan2,cay - cs,
    cax + L,cay - cs);

  // top
  p.bezierCurveTo(cax + L - tan1,cay - cs,
    cax + tan2,cay - cs,
    cax,cay - cs);

  // upper left
  p.bezierCurveTo(cax - tan1,cay - cs,
    cax - cs,cay - tan2,
    cax - cs,cay);

  // lower left to close
  p.bezierCurveTo(cax - cs,cay + tan1,
    cax - tan2,cay + cs,
    cax,cay + cs);

  // Rotate the entire thing
  let a = -12 * Math.PI / 180; // angle to rotate
  let p2 = p.rotateAbout(a,new Point2D(cax,cay));
  
  ctx.lineWidth = 2.0;

  p2 = p2.reflectX();
  p2 = p2.translate(new Point2D(0,120));

  ctx.stroke(p2);
}
\end{figput}
\caption{Basic Constraints on Gears}
\label{fig-basic-gear-constraints}
\end{figure}

See Figure (\ref{fig-basic-involute}). 
Imagine a string wrapped around the circle, with one end fixed to the
circle and a pencil at the other end. As the string unwraps from the
circle, the pencil traces out a 
curve called the involute. The line determined by the string is
obviously tangent to the circle at the point at which it meets the circle. The
line is also perpendicular to the involute because, at each
instantaneous point of rotation, the involute is locally an arc of the
circle formed by the string. For a circle of radius $r$ centered at the
origin, the involute can be parameterized by $i(t)=(i_x(t),i_y(t))$, where
\begin{eqnarray*}
i_x(t) &=& r (\cos t + t \sin t) \\
i_y(t) &=& r (\sin t - t \cos t)
\end{eqnarray*}
As we will see, the notable thing about the involute is that when gear
teeth take the form of an involute, the pitch point is constant
throughout the gears' motion. 

\begin{figure}[b!]
\begin{figput}{basicinvolute,140bp}[done,skip]
// An extra function to define the involute.
function invo(t) {
  
  // Basically, if you have a circle of radius a, the parametric form is
  // x(t) = a (cos t + t sin t)
  // y(t) = a (sin t - t cos t)
  // The parameterization starts with t = 0, then runs as an angle in radians.

  // These values copied from below.
  let cx = 160;
  let cy = 52;
  let a = 50;

  let c = Math.cos(t);
  let s = Math.sin(t);

  return new Point2D( cx + a*(c + t*s), cy + a*(s - t*c));  
}

function basicinvolute(ctx) {

  // A looping animation that draws an involute. The string unwraps
  // and re-wraps in a loop. Note the doubling of numSteps since I
  // want back-and-forth action.
  let numSteps = 100;
  let loopw = LoopAnimWidget.register(ctx,-30,55,0.85,
  true, // visible
  2*numSteps, // steps per loop
  0, // starting step
  20, // ms time step
  //false,false,false,false,false, // nothing is visible
  true,true,true,true,true, // Everything visible
  'loop');

  let cx = 160;
  let cy = 52;
  let r = 50;

  // circle
  let p = new FPath();
  p.ellipse(cx,cy,r,r,0,0,2*Math.PI);
  ctx.strokeStyle = 'black';
  ctx.lineWidth = 0.4;
  ctx.stroke(p);

  // And a dot at the center.
  p = new FPath();
  p.ellipse(cx,cy,2,2,0,0,2*Math.PI);
  ctx.fillStyle = 'black';
  ctx.fill(p);

  // Draw the involute by converting to a bezier.
  // The level of ``unwrap'' is determined by the number of steps of
  // the animation.
  let t = 2 * Math.abs((loopw.curStep - numSteps) / numSteps);  
  let segs = Math.ceil(15 * t);
  p = FPath.parametricToBezier(invo,0,t,segs);
  ctx.lineWidth = 1.2;
  ctx.stroke(p);

  // The string goes from the end of the involute to a point tangent
  // to the circle. So, I have a circle centered at
  // (c_x,c_y) and a point p = (p_x,p_y) = end-point of involute, and
  // I want the point on the circle such that the line from the point on
  // the circle to p is tangent to the circle. The circle is parameterized
  // by (c_x,c_y) + r(cos a,sin a), (a varies) so the tangent to the circle is
  // parameterized by r(-sin a,cos a). I need such a tangent vector that points
  // to p. Let t(a) = r(-sin a,cos a) and c(a) = r(cos a,sin a). I need to
  // find a such that c(a) + s t(s) = p, for some s. Thus, I need to solve
  // (c_x + r cos a - s r sin a,c_y + r sin a + s r cos a) = (p_x,p_y)
  // for a and s (although I don't really care about s). We have
  // c_x + r cos a - s r sin a = p_x implies
  // s = ( c_x + r cos a - p_x) / r sin a
  // Plugging into the other coordinate, we get
  // p_y = c_y + r sin a + s r cos a
  // p_y = c_y + r sin a + r cos a ( c_x + r cos a - p_x) / r sin a
  // p_y = c_y + r sin a + cos a ( c_x + r cos a - p_x) / sin a
  //
  // BLAH. That is a mess. Easier is to do it by construction.
  // We have a circle centered at C and a point P. Let D be the midpoint
  // of line line CP. The circle about D passing through C and P meets the
  // circle around C at two points, E and F. The line from C to E (or to F)
  // is a right angle since the angle is ``inscribed'' in the circle about D.
  // This is a general fact, that if you take two points that are opposite
  // each other in a circle (C and P in my example) and *any* other point
  // on that circle, then the angle formed by the two opposite points and
  // that other point is a right angle.
  // If that is a right angle, then the line from P to E (or to F) is tangent
  // to the circle.
  //
  // Moral of the story...We have circle centered at (c_x,c_y) and p=(p_x,p_y),
  // the end-point of involute. The line from p tangent to the circle is
  // obtained by defining d=(d_x,d_y) as the midpoint of c and p. The
  // intersection of the circle around d with the circle around c is the
  // point in question.
  //
  // I've reduced to the problem of finding the points of intersection of two
  // circles. We have
  // (x-c_x)^2 + (y-c_y)^2 = r_c^2
  // (x-d_x)^2 + (y-d_y)^2 = r_d^2
  // Subtract the second from the first to obtain
  // (x-c_x)^2 - (x-d_x)^2 + (y-c_y)^2 - (y-d_y)^2 = r_c^2 - r_d^2
  // Expand and collect:
  // x^2-2xc_x+c_x^2 - x^2+2xd_x-d_x^2 + y^2-2yc_y+c_y^2 - y^2+2yd_y-d_y^2 =
  //     r_c^2 - r_d^2
  // -2xc_x+c_x^2 +2xd_x-d_x^2 - 2yc_y+c_y^2 +2yd_y-d_y^2 = r_c^2 - r_d^2
  // 2x(d_x-c_x) + 2y(d_y-c_y) + c_x^2 -d_x^2 +c_y^2 -d_y^2 = r_c^2 - r_d^2
  // 2x(d_x-c_x) + 2y(d_y-c_y) = (r_c^2-r_d^2) + (c_x^2-d_x^2) + (c_y^2 -d_y^2)
  // Solve for y (say):
  // y = x(c_x-d_x) / 2(d_y-c_y) +
  //    [ (r_c^2-r_d^2) + (c_x^2-d_x^2) + (c_y^2 -d_y^2) ] / 2(d_y-c_y)
  // and plug this back into the original equation for the circle and you
  // can solve for x. First, write the above eqn for y as
  // y = x(c_x-d_x) / D + r_m / D
  //   = [ x(c_x-d_x) + r_m ] / D
  // where r_m means ``r-mess'' and D is for ``denominator.''
  // (x-c_x)^2 + (y-c_y)^2 = r_c^2
  // (x-c_x)^2 + [ ( x(c_x-d_x)+ r_m ) / D - c_y ]^2 = r_c^2
  // (x-c_x)^2 + ( x(c_x-d_x)+ r_m )^2 / D^2 - c_y ]^2 = r_c^2
  // Etc. In theory, this is a quadratic in x, but wow.
  //
  // That is another mess. Back up a bit, and consider the general case
  // of the intersection of two circles, c and d.
  // Let D = distance between centers and P be the point of intersection.
  // Draw a line between the centers and two triangles, one from center c to
  // P and then perpendicular to the line betwene centers, and the other
  // triangle similar, but in circle d. Let E be the point where the
  // perpendicular leg of the triangles intersects the line between centers.
  // We have distances a and b, a from center of C to E, and b from center
  // of d to E. Let h be the height of this perpendicular leg. We have
  // D = a + b, a^2 + h^2 = r_c^2 and b^2 + h^2 = r_d^2. Solve for a and b.
  // This is really the same is the big messy thing, but easier to follow.
  //
  // FORGET ABOUT THE ABOVE. THIS IS THE WAY TO DO IT.
  // Back way up to the very begining and reconsider the whole thing with
  // simpler assuptions. Suppose that the circle is centered at the origin,
  // so we can ignore (c_x,c_y). Assume also that the point is on the
  // x-axis so that p = (p_x,p_y) = (p_x,0). Now the circle is
  // C(a) = r (cos a, sin a)
  // C'(a) = r (-sin a, cos a).
  // We want the line through C(a) pointing in the C'(a) direction to
  // pass through p. That is, we require a solution (in t and a) to
  // C(a) + t C'(a) = p
  // r cos a - r t sin a = p_x and r sin a + r t cos a = 0.
  // So t = - sin a / cos a = - tan a. Then
  // r cos a + r sin^2 a / cos a = p_x
  // r cos^2a + r sin^2 a = p_x cos a
  // r = p_x cos a
  // THAT we can solve for a. We get this solution, but then have to shift
  // and rotate. Rotate the entire plane by the angle that p is relative to
  // the x-axis. Then shift to move the origin of the circle to (c_x,c_y).
  //
  // BUG: This entire mess belongs in some geometry module, preferably with
  // a complete discussion of the math.

  let P = invo(t);

  // This is p_x in the analysis above.
  let Pd = Math.sqrt((P.x - cx)**2 + (P.y - cy)**2);

  // There are two possible solutions for r = p_x cos a or
  let a = Math.acos(r/Pd);

  // Now have a solution that's valid for P on the x-axis. Rotate (Pd,0) to P.
  // This has the effect of rotating C(a) (with the a just determined this
  // would be the tangent point if P were on the x-axis and C were centered
  // at the origin) through an angle b, determined by tan b = P_x/P_y.
  // But we want this relative to (cx,cy), so shift.
  let b = Math.atan2(P.y-cy,P.x-cx);

  // The point we want is now C(a+b), then translated by (cx,cy).
  let E = new Point2D(r*Math.cos(a+b)+cx,r*Math.sin(a+b)+cy);
  
  p = new FPath();
  p.moveTo(P.x,P.y);
  p.lineTo(E.x,E.y);
  ctx.lineWidth = 0.8;
  ctx.stroke(p);

}
\end{figput}
\caption{The Involute of a Circle}
\label{fig-basic-involute}
\end{figure}

Figure (\ref{fig-constant-pp}) shows two disks acting as gears by
simple friction. The circle of each such disk is 
called the \emph{pitch circle} (where they come in contact is the
pitch point). The centers of these disks are joined 
by the \emph{line of centers}, and the distance between these centers
is the \emph{center distance}. Now imagine two slightly smaller and
concentric circles, called the \emph{base circles}. These base circles
will be used to form involutes, and these involutes will be the
profiles of the gear teeth.


\begin{figure}
\begin{figput}{constpp,140bp}[done,skip]
function constpp(ctx) {

  // A static figure, no interaction.
  let rc = 60;
  let rd = 40;
  let y = 70;
  
  let c = new Point2D(100,y);
  let d = new Point2D(c.x + rc+rd,y);

  // Pitch circles
  let p = new FPath();
  p.ellipse(c.x,c.y,rc,rc,0,0,2*Math.PI);
  ctx.strokeStyle = 'black';
  ctx.lineWidth = 1.5;
  ctx.stroke(p);

  p = new FPath();
  p.ellipse(d.x,d.y,rd,rd,0,0,2*Math.PI);
  ctx.strokeStyle = 'black';
  ctx.lineWidth = 1.5;
  ctx.stroke(p);

  // Base circles
  let s = 0.80;
  p = new FPath();
  p.ellipse(c.x,c.y,s*rc,s*rc,0,0,2*Math.PI);
  ctx.strokeStyle = 'black';
  ctx.lineWidth = 0.4;
  ctx.stroke(p);

  p = new FPath();
  p.ellipse(d.x,d.y,s*rd,s*rd,0,0,2*Math.PI);
  ctx.strokeStyle = 'black';
  ctx.lineWidth = 0.4;
  ctx.stroke(p);

  ctx.font = '10px san-serif';
  ctx.fillStyle = 'black';
  
  drawText(ctx,'base circle', c.x,c.y,-4,0);
  drawText(ctx,'pitch circle', d.x,c.y+rd+6,0,0);
}
\end{figput}
\caption{Base Circle and Pitch Circle}
\label{fig-constant-pp}
\end{figure}


\begin{figure}[b!]
\begin{figput}{moveinvo,200bp}[done,skip]
// At one point, I was making a small manual adjustment on the figure
// placement here. I did this by wrapping all the drawing
// (after the \useasbounding box) with
// \begin{scope}[shift={(40bp,0bp)}]
// then \end{scope} just before \end{tikzpicture}
  
function moveinvo(ctx) {

  // Similar to above, but don't draw the pitch circles, and only a part
  // of the base circles is drawn, and they're larger. There is also
  // a line for the line of action and an animation to show what the
  // point of contact does over time. OK, so not that similar!

  let numSteps = 1000;
  let loopw = LoopAnimWidget.register(ctx,-80,100,1.0,
    true, // not hidden
    numSteps, // steps per loop
    0, // starting step
    10, // ms time step
    true,true,true,true,true,
    'loop');

  let rc = 80;
  let rd = 60;
  let y = 100;
  let s = 35;
  
  let c = new Point2D(90,y);
  let d = new Point2D(c.x + rc+rd +s,y);

  // Base circles
  p = new FPath();

  // I played around, trying to show only portions of the disks so as to
  // save space, but it doesn't look right. In some ways, that
  // simplifies things.
  p.ellipse(c.x,c.y,rc,rc,0,0,2*Math.PI);
  ctx.strokeStyle = 'black';
  ctx.lineWidth = 1.0;
  ctx.stroke(p);
  
  p = new FPath();
  p.ellipse(d.x,d.y,rd,rd,0,0,2*Math.PI);
  ctx.strokeStyle = 'black';
  ctx.lineWidth = 1.0;
  ctx.stroke(p);

  /* I was going to just draw these arcs, but better to do the
  // whole cirles.
  //In fact, defer drawing this. I want the entire wedge to turn.
  // This arc is centered at (0,0).
  let cVisAngle = 4*Math.PI/8;
  p = FPath.arcToBezierNEW(rc,-cVisAngle,cVisAngle);
  p = p.translate(c);
  ctx.lineWidth = 2.0;
  ctx.stroke(p);
  
  let dVisAngle = 5*Math.PI/8;
  p = FPath.arcToBezierNEW(rd,dVisAngle,-dVisAngle);
  p = p.translate(d);
  ctx.lineWidth = 2.0;
  ctx.stroke(p);
  */

  // The line of action.
  // This involves another calculation of tangent lines. See Graphics Gems I,
  // p. 7 and Figure 8. Use similar triangles. We have two distances,
  // x_c and x_d from the respective centers of the circles to the point
  // where the desired line intersects the line connecting the centers.
  // We have x_c + x_d = D = distance between centers, and by
  // similar triangles r_c/x_c = r_d/x_d. A little algebra gives
  // x_c = r_c D / (r_c + r_d).
  //
  // We now have one point that must be on the line, and we go through the
  // same rigmarole as above to find the line tangent to a circle. The only
  // difference is that we've chosen the point through which the circle must
  // pass in such a way that the resulting line will be tangent to both
  // circles if it's tangent to either of them.
  //
  // As above, make the simplifying assumption that circle C is centered
  // at the origin. Then do a rotate and translate at the end.
  let D = Math.sqrt((c.x-d.x)**2 + (c.y-d.y)**2);
  let xc = rc * D / (rc + rd);

  // We are working (for now) as though circle C is centered at the origin.
  // So this is the point we want.
  let M = new Point2D(xc,0);

  // As in the earlier case, where I did the involute, we get the tangent.
  // The difference is that the point is now M, not P, as it was above.
  let Md = Math.sqrt((M.x - c.x)**2 + (M.y - c.y)**2);
  let a = Math.acos(rc/Md);
  let b = Math.atan2(M.y,M.x);
  let E = new Point2D(rc*Math.cos(-a+b)+c.x,rc*Math.sin(-a+b)+c.y);
  
  // The above draws from circle C to the central point M, but I need to
  // extend the line all the way to the other circle. So I need to know where
  // this line intersects that circle. Recall that M was found by similar
  // triangles. The angle of the tangent point relative to the circles' centers
  // is the same in both cases, although for the circle on the right, we
  // have to adjust by pi.
  b += Math.PI;
  let F = new Point2D(rd*Math.cos(-a+b)+d.x,rd*Math.sin(-a+b)+d.y);

  p = new FPath();
  p.moveTo(E.x,E.y);
  p.lineTo(F.x,F.y);
  ctx.lineWidth = 0.6;
  ctx.stroke(p);
  
  // The above draws the more or less static setup. Now I want to draw a
  // particular point on the line of action, and let it move as the disks
  // rotate, tracing out a pair of involutes.
  //
  // Use M for the particular point along the line, which is detemined
  // by t, using a linear parameterization of the line E to F.
  // This value is what determines the frame of the animation.
  let t;
  if (loopw.curStep < numSteps/2)
    t = 2* loopw.curStep/numSteps;
  else
    t = 2 *(numSteps - loopw.curStep)/numSteps;
  
  M = new Point2D(E.x + t*(F.x-E.x),E.y + t*(F.y-E.y));

  p = new FPath();
  p.ellipse(M.x,M.y,1.5,1.5,0,0,2*Math.PI);
  ctx.fillStyle = 'black';
  ctx.fill(p);

  // I need to know the total size of the tooth. So I need the value for
  // t at which the tooth will barely clear the opposing gear. The value
  // I am looking for has nothing to do with the current point of contact,
  // and is always the same (which is sort of the point). Scale down the
  // entire picture and find t for which the distance of involute(t) from
  // the center of the left disk is equal to the distance between the
  // edges of the two disks.
  //
  // BUG: A common problem is that lines have thickness. If you want line A
  // to barely touch line B, then thicknesses matter. The strictly correct way
  // to deal with this is probably to treat lines as filled rectangles (or
  // quadrilaterals more generally). This general philisophy is one way to
  // handle stuff like arrowheads. Another way might be to allow stroking
  // the line on one side of its geometric definition or the other.
  // I may have mentioned this issue before.
  // 
  // See above. rc and rd are the two disks' radii and the centers at c and d.
  // The distance we care about is d.x-c.x-rd before scaling. Subtract off
  // a hair to take into account line thicknesses. Scaling then gives:
  let toothD = (d.x-c.x-rd - 2) / rc;

  // And the value that brings the unit involute out to this distance.
  let toothT = Numerical.newton(solveToothD,0.5,0,3,toothD,0.00001);

  // For circle C, we have an involute extending out to M.
  // The standard parameterization of the involute assumes that the point
  // starts at the x-axis, and that is not true here. The ``point of initial
  // contact'' with the base circle varies, and that is sort of the point
  // for gear design. At any rate, the usual parameterization is
  // x(a) = r (cos a + a sin a)
  // y(a) = r (sin a - t cos a)
  // What we need to do is to add some constant to a so as to shift
  // the involute -- effectively rotate it around the center of the circle.
  //
  // The distance from M to E is how much string wraps around the circle C.
  // This determines the point of initial contact. Let d_M = |M-E|.
  // When wrapped around the circle, this corresponds to an angle d_M/r_c. 
  // It wraps from the point E, and that point is at angle -a+b relative to
  // the x-axis. So the adjustment we need to make is a_M = d_M/r_c +a-b.
  // There's also a pi adjustment due to LH coordinates.
  //
  // In addition, we need to know at what value of a (as argument to the
  // parameterization) the involute reaches M. THAT is not so easy since it
  // amounts to inverting the involute. We want to find t such that
  // involute(t) = M, where M is the point in question. Assuming that (u,v)
  // is on the involute, and considering the ``standard'' involute, what
  // we want is t such that
  // u = cos t + t sin t
  // v = sin t - t cos t
  // I don't think there's any good way to do this, and a bit of googling
  // bears that out. I'm not going to try to be mathematically clever.
  // Just use Newton's method. So, take the point, M, and "undo" the
  // transformation to get M relative to the standard involute -- I mean for
  // a unit circle at the origin.
  let dM = Math.sqrt((M.x-E.x)**2 + (M.y-E.y)**2);
  let aM = dM/rc - a + b - Math.PI;

  // Untranslate, unscale and unrotate M.
  // BUG: This foolishness is why I should have worked with the standard unit
  // involute throughout. 
  let unitM = new Point2D(M.x - c.x,M.y-c.y);
  unitM = new Point2D(unitM.x/rc,unitM.y/rc);
  unitM.rotateSelf(-aM);

  // Now I want t such that invo2(t).x = unitM.x. I'm using x instead of
  // y since y varies very little near the circle and I'm afraid that
  // the root finder won't converge well.
  // Note that t = 3 is roughly where the derivative changes sign.
  // That's why it's a good place for bracketing.
  t = Numerical.newton(solveInvo,0.5,0,3,unitM.x,0.00001);

  // However, we do NOT want to use a value for t larger than toothT
  // since the tooth would make the tooth magically grow to touch M.
  if (t > toothT) t = toothT;
  
  p = FPath.parametricToBezier(invo2,0,t,30);
  p = p.rotate(aM);
  p = p.scale(rc);
  p = p.translate(c);

  ctx.lineWidth = 1.0;
  ctx.stroke(p);

  // Now, exactly as above, draw a bit of the involute.
  p = FPath.parametricToBezier(invo2,t,toothT,30);
  p = p.rotate(aM);
  p = p.scale(rc);
  p = p.translate(c);
  
  ctx.lineWidth = 0.4;
  ctx.stroke(p);
  
  // And...what the heck... some spokes to make the rotation
  // more noticable.
  p = new FPath();
  p.moveTo(c.x,c.y);
  let sp = new Point2D(c.x+rc,c.y);
  sp = sp.rotateAbout(c,aM);
  p.lineTo(sp.x,sp.y);
  ctx.lineWidth = 0.4;
  ctx.stroke(p);

  // let wedgeRange = cVisAngle / 4;
  let wedgeRange = Math.PI / 6;
  for (let i = 0; i < 10; i++)
    {
      let a = aM + i*wedgeRange;
      //if ((a <= cVisAngle) && (a >= -cVisAngle))
        {
          p = new FPath();
          p.moveTo(c.x,c.y);
          let sp = new Point2D(c.x+rc,c.y);
          sp = sp.rotateAbout(c,a);
          p.lineTo(sp.x,sp.y);
          ctx.lineWidth = 0.4;
          ctx.stroke(p);
        }

      a = aM - i*wedgeRange;
      //if ((a <= cVisAngle) && (a >= -cVisAngle))
        {
          p = new FPath();
          p.moveTo(c.x,c.y);
          let sp = new Point2D(c.x+rc,c.y);
          sp = sp.rotateAbout(c,a);
          p.lineTo(sp.x,sp.y);
          ctx.lineWidth = 0.4;
          ctx.stroke(p);
        }
    }

  // Now we have to do almost the same thing, but for the right disk.
  // The twist is that the involute is now flipped relative to the
  // earlier case because the right disk rotates in the opposite direction.
  // That is, swap the sign of the y-coordinate, and the x-coordinate too
  // since the disk is on the right.
  //
  // The tooth height is the same on the right side, but the scale is different,
  // so the value of toothT will be different.
  toothD = (d.x-c.x-rc - 2) / rd;
  toothT = Numerical.newton(solveToothD,0.5,0,3,toothD,0.00001);
  
  // What we want is for the flipped involute to pass through M, although
  // it will pass through M at a different value for t. I could work out
  // this t by finding it numerically. Or, I could use the fact that --
  // because I know how gears work -- the relative rates of rotion are known.
  // In fact, because I want the line to change at M, I do need this value
  // for t. This is more like what happens physically: the motion of the
  // left disk determines M, which pushes the disk on the right to rotate.
  //
  // Redetermine unitM, as above, but relative to the right disk.
  // Do it afresh to avoid any rounding error.
  dM = Math.sqrt((M.x-F.x)**2 + (M.y-F.y)**2);
  aM = dM/rd - a + b - Math.PI;
  unitM = new Point2D(d.x-M.x ,d.y-M.y);
  unitM = new Point2D(unitM.x/rd,unitM.y/rd);
  unitM.rotateSelf(-aM);

  // This is the value for t at which the RH involute passes through M.
  t = Numerical.newton(solveInvo,0.5,0,3,unitM.x,0.00001);
  if (t > toothT) t = toothT;

  // Path from disk to M
  p = FPath.parametricToBezier(invo2,0,t,30);
  p = p.reflectXY();
  p = p.rotate(aM);
  p = p.scale(rd);
  p = p.translate(d);
  
  ctx.lineWidth = 1.0;
  ctx.stroke(p);

  // Path from M to the tip of the tooth.
  p = FPath.parametricToBezier(invo2,t,toothT,30);
  p = p.reflectXY();
  p = p.rotate(aM);
  p = p.scale(rd);
  p = p.translate(d);
  
  ctx.lineWidth = 0.4;
  ctx.stroke(p);

  // And some spokes for visual appeal, as above.
  // The outer arc first.
  /*
  dVisAngle = 5*Math.PI/8;
  p = FPath.arcToBezier(rd,dVisAngle,-dVisAngle);
  p = p.rotate(aM);
  p = p.translate(d);
  ctx.lineWidth = 2.0;
  ctx.stroke(p);
  */
  
  // And the spokes.
  p = new FPath();
  p.moveTo(d.x,d.y);
  sp = new Point2D(d.x-rd,d.y);
  sp = sp.rotateAbout(d,aM);
  p.lineTo(sp.x,sp.y);
  ctx.lineWidth = 0.4;
  ctx.stroke(p);

  // In fact, the angle *spanned* by the arc is two times this.
  // It's because of the whole LH/RH thing.
  for (let i = 0; i < 7; i++)
    {
      let a = aM + i*wedgeRange;
      //if ((a <= dVisAngle) && (a >= -dVisAngle))
        {
          p = new FPath();
          p.moveTo(d.x,d.y);
          let sp = new Point2D(d.x-rd,d.y);
          sp = sp.rotateAbout(d,a);
          //sp.x = d.x - (sp.x - d.x);
          p.lineTo(sp.x,sp.y);
          ctx.lineWidth = 0.4;
          ctx.stroke(p);
        }

      a = aM - i*wedgeRange;
      p = new FPath();
      p.moveTo(d.x,d.y);
      let sp = new Point2D(d.x-rd,d.y);
      sp = sp.rotateAbout(d,a);
      p.lineTo(sp.x,sp.y);
      ctx.lineWidth = 0.4;
      ctx.stroke(p);
    }
}

function solveToothD(t) {
  let p = invo2(t);
  return Math.sqrt(p.x**2 + p.y**2);
}

function solveInvo(t) {
  return invo2(t).x;
}

function invo2(t) {

  // Involute, assuming circle of radius 1 centered at the origin.
  // There is another function called invo(), hence this is invo2().
  // BUG: This is the way I should have done this above. Generate a standard
  // involute, then scale and rotate as needed.
  let c = Math.cos(t);
  let s = Math.sin(t);

  // Note that y is backwards since LH coordinate system.
  return new Point2D( c + t*s , - (s - t*c) );
}
\end{figput}
\caption{Rotation Traces Involutes}
\label{fig-moving-involute}
\end{figure}


Figure (\ref{fig-moving-involute}) shows an enlarged view of the two
base circles of Figure (\ref{fig-constant-pp}), without the pitch
circles. Imagine that a piece of string is tightly
wrapped around one base circle, extends over to the other base circle,
and is wrapped around it too. As the disks rotate, the string unwinds
from one base circle and is taken up by the other base circle. There
is a fixed point on the string that represents the point of
contact between two teeth. This point traces a path that
is an involute relative to either circle. These involutes define the
shape of the mating tooth profiles. The line of action is coincident
with the string, and the two gears have conjugate action.

A fortunate feature of the involute is that the teeth can be truncated at
their perimeter, or their widths may be varied, yet the two gears
still have conjugate action; the teeth come into contact sooner or
later as they rotate, but the point of contact follows the same line
of action. If the center distance changes, then the line of action
also changes, but the tooth form (the involute) remains the same, and
the gears still have conjugate action, though there will be some
backlash and additional friction between the teeth.

What remains is the resolution of many practical issues: the interplay
of gear radius, tooth size, number of teeth and the
like.

The two main methods of gear specification are
metric (ISO) and inch (AGMA), and the two systems use slightly different
fundamental quantities to specify a given gear. These are the basic
parameters used to specify off-the-shelf gear profiles.
\begin{small}
\begin{itemize}[itemsep=-1pt]
\item $\phi$, pressure angle
\item $N$, number of teeth
\item $m$, module (for metric gears)
\item $p_d$, diametral pitch (for inch gears)
\end{itemize}
\end{small}

There are many additional terms and measurements. In fact,
off-the-shelf gears are typically specified in a way that makes
various assumptions. Additional parameters that influence 
gear design are 
\begin{small}
\begin{itemize}[itemsep=-1pt]
\item $p_c$, circular pitch
\item $d$, pitch diameter
\item $r_p$, pitch radius
\item $r_b$, base radius
\item $a$, addendum
\item $b$, dedendum
%\item $C$, center distance
%\item $r$, pitch radius
\end{itemize}
\end{small}

The \emph{pressure angle}, $\phi$, is the angle between the line of
action and the perpendicular to the line of centers. In Figure
(\ref{fig-moving-involute}), the pressure angle is roughly $35^\circ$.
If two gears are to mesh without backlash, then they must use the same
pressure angle. At one time, $14.5^\circ$ was a commonly used pressure
angle, but $20^\circ$ is the current standard.
A more obviously important choice is the number of teeth, $N$. Since
the number of teeth determines the ratio of any gear train, this is a crucial
choice, but it raises the question of how to fit $N$ teeth on a given
gear.

The tooth-to-tooth distance, as measured along the arc of the pitch
circle, is the \emph{circular pitch}, $p_c$, and the corresponding
diameter is the \emph{pitch diameter}, $d$. Since $N p_c$ is the
circumference of the pitch circle, we have
$$p_c = {\pi d\over N}.$$
This is the inches or mm per tooth, measured along the circumerence. In
practice, AGMA gears are specified by the \emph{diametral pitch}, 
$$p_d = {N\over d}.$$
This is the teeth per $\pi$ inches, and seems like an odd
choice, but that's how it's done.
Metric gears are specified by their \emph{module}, $m$, which is
stated in millimeters, and is
$$m = {d\over N} = {1\over p_d}.$$
The actual tooth-to-tooth distance is thus $\pi m$. 
The values around which the two systems, ISO and AGMA,
are standardized are not compatible. For example, a module of $m=4$
corresponds to a diametral pitch of
$$p_d = {25.4\over 4} = 6.35,$$
which is not a standard AGMA size.

The profile of each gear tooth is an involute, and determining the
involute requires that the base circle be known. See Figure
(\ref{fig-base-from-pitch}), in which the outer circle is the pitch circle and
the inner circle is the base circle. Let $r_p$ be the radius of the
pitch circle, and $r_b$ be the radius of the base circle. Because the
angle determined by where the line of action meets the base circle is
equal to the pressure angle, $\phi$, we have
$$r_b = r_p \cos\phi.$$

\begin{figure}[b!]
\begin{figput}{baseandpitch,175bp}[done,skip]
function baseandpitch(ctx) {

  // A static figure, no interaction.
  let rc = 70;
  let y = 80;

  let c = new Point2D(130,y);

  // Pitch circle
  let p = new FPath();
  p.ellipse(c.x,c.y,rc,rc,0,0,2*Math.PI);
  ctx.strokeStyle = 'black';
  ctx.lineWidth = 0.4;
  ctx.stroke(p);

  // Radius through pitch point
  let pplen = rc + 50;
  p = new FPath();
  p.moveTo(c.x,c.y);
  p.lineTo(c.x+pplen,c.y);
  ctx.stroke(p);
  
  // Line of action through P and angle phi. Use tan phi = dx / dy
  // to draw a line through P = (c.x+rc,c.y).
  let phi = Math.PI / 6;
  let lenup = 70;
  let lendown = 50;
  let tan = Math.tan(phi);
  p = new FPath();
  p.moveTo(c.x + rc - lenup * tan,c.y + lenup);
  p.lineTo(c.x + rc + lendown * tan,c.y - lendown);
  ctx.stroke(p);
  
  // Base circle has radius:
  let rb = rc * Math.cos(phi);
  p = new FPath();
  p.ellipse(c.x,c.y,rb,rb,0,0,2*Math.PI);
  ctx.strokeStyle = 'black';
  ctx.lineWidth = 0.4;
  ctx.stroke(p);

  // Vertical line through P.
  let vlen = 60;
  p = new FPath();
  p.moveTo(c.x+rc,c.y-vlen);
  p.lineTo(c.x+rc,c.y+vlen);
  ctx.stroke(p);

  // radial line to where line of action meets base circle.
  p = new FPath();
  p.moveTo(c.x,c.y);
  p.lineTo(c.x + rb*Math.cos(phi),c.y+rb*Math.sin(phi));
  ctx.stroke(p);

  // Little hash marks to indicate angles have same measure.
  p = FPath.circArcToBezier(16,0,phi);
  p = p.translate(c);
  ctx.stroke(p);
  
  p = FPath.circArcToBezier(16,Math.PI/2 - phi,Math.PI/2);
  p = p.reflectX();
  p = p.translate(new Point2D(c.x+rc,c.y));
  ctx.stroke(p);
}

\end{figput}
\caption{Base Circle from Pitch Circle}
\label{fig-base-from-pitch}
\end{figure}


The module or diametral pitch determines the tooth-to-tooth distance,
but it doesn't determine how much of that space is solid tooth and how
much is the space between teeth. The ensure that there is no backlash,
the thickness of each tooth, as measured along the arc of the pitch
circle, should be equal to the space between teeth -- for practical
reasons (lubrication), the gap between teeth is often made one or two 
thousandths of an inch wider than this. Again, these distances are
\emph{as measured along the arc of the pitch circle}. In practice, it
is easier to work with the angles subtended by these arcs.

There is one further issue to resolve. See Figure (\ref{fig-teeth-touch}),
which shows a portion of a base circle and slightly larger pitch
circle, with an involute. What's needed is the measure of the small
angle relative to the $x$-axis at which the involute meets the pitch
circle. The involute is parametrized by 
$$i(t) = r_b(\cos t + t\sin t,\sin t - t\cos t) = (i_x(t),i_y(t)),$$
and the involute meets the pitch circle when $|i(t)|^2 = r_p^2$. We have
\begin{eqnarray*}
|i(t)|^2 &=& r_b^2\left[(\cos t + t\sin t)^2 + (\sin t - t\cos t)^2\right] \\
&=& r_b^2\left[\cos^2 t + 2t\cos t\sin t + t^2\sin^2t + 
  \sin^2 t - 2t\cos t\sin t + t^2\cos^2 t\right] \\
&=& r_b^2(1 + t^2),
\end{eqnarray*}
and $|i(t)| = r_p$ implies that
$$t = \sqrt{\left({r_p\over r_b}\right)^2-1} = \sqrt{\left({r_p\over
    r_p\cos\phi}\right)^2-1}\  = \tan\phi.$$
The angle made by the line through $i(t)$ with the $x$-axis is
$\alpha$, where\footnote{There
  seems to be no standard notation for this angle. In fact, I have
  found no mention of this issue in any common reference, even though it's crucial for
  determining the profile.} 
$$\tan\alpha = i_y(t)/i_x(t).$$

\begin{figure}
\begin{figput}{teethtouch,100bp}[done,skip]
function teethtouch(ctx) {

  let c = new Point2D(100,20);
  let rb = 80;

  // Bit of arc on the right. This is the base circle.
  let span = Math.PI/5;
  let p = FPath.circArcToBezier(rb,0,span);
  //p = p.reflectX();
  p = p.translate(c);
  ctx.lineWidth = 1.0;
  ctx.stroke(p);

  // And a slightly larger arc for the pitch circle.
  let rp = rb + 22;
  p = FPath.circArcToBezier(rp,0,span);
  //p = p.reflectX();
  p = p.translate(c);
  ctx.lineWidth = 0.4;
  ctx.stroke(p);
    
  // Draw x-axis/radius.
  let width = 150
  p = new FPath();
  p.moveTo(c.x,c.y);
  p.lineTo(c.x+width,c.y);
  ctx.lineWidth = 0.4;
  ctx.stroke(p);
  
  // Using the earlier invo2() function for the unit involute.
  let a = Math.PI/8;
  let inv = FPath.parametricToBezier(invo2,0,1.5,30);
  p = inv.reflectX();
  //p = inv.scale(rb);
  p = p.scale(rb);
  p = p.translate(c);
  ctx.stroke(p);

  // And a line out to the point where the top involute meets the pitch circle.
  p = new FPath();
  p.moveTo(c.x,c.y);
  let t = Math.sqrt((rp/rb)**2 - 1);
  let q = invo2(t);
  q = new Point2D(q.x,-q.y);
  q = new Point2D(q.x*rb,q.y*rb);
  q = new Point2D(q.x+c.x,q.y+c.y);
  p.lineTo(q.x,q.y);
  ctx.lineWidth = 0.4;
  ctx.stroke(p);
}
\end{figput}
\caption{Angle Subtended by Involutes.}
\label{fig-teeth-touch}
\end{figure}

We now have enough information to begin laying out gear
profiles. Suppose that $\phi$, $N$ and $m$ (or $p_d$) are given. There
will be $N$ involutes runing one way, and $N$ running the other
way. Relative to the base circle, the tooth-to-tooth distance subtends
an angle measuring $2\pi/N$. Half of this is solid tooth, and half is the
gap between teeth, with an adjustment for $\alpha$. So each solid
tooth subtends the angle $\pi/N + 2\alpha$, and each gap
subtends the angle $\pi/N - 2\alpha$. Figure~(\ref{fig-gear-example})
shows the result for $\phi = 20^\circ$, $N=15$ and $m =4$. 

\begin{figure}
\begin{figput}{gearexample,120bp}[done,skip]
function gearexample(ctx) {

  // Note that I may have written this for a LH coordinate system, but
  // it doesn't matter since it's radially symmetric.
  
  // Pressure angle, number of teeth and module.
  let phi = 20 * Math.PI / 180;
  let N = 15;
  let m = 5;

  let c = new Point2D(175,60);
  let pitchDiam = N * m;

  // Pitch radius and base radius.
  let rp = pitchDiam / 2;
  let rb = rp * Math.cos(phi);

  ctx.lineWidth = 0.4;

  // Point along parameterization where involute meets pitch circle.
  let t = Math.tan(phi);

  // Adjustment to width of the base of the teeth due to the fact that
  // the teeth are of varying width. Note that the radius of the base circle
  // doesn't matter.
  let ipt = invo2(t);
  let alpha = -Math.atan2(ipt.y,ipt.x);

  ctx.lineWidth = 0.4;

  // Each tooth, at the base, takes 2pi/2N + 2 alpha. The space between
  // teeth is 2pi/2N - 2 alpha
  //
  // We want the x-axis to split a gap between teeth, so the first tooth
  // starts at (pi/N - 2 alpha) / 2 = pi/2N - alpha.
  //
  // Using earlier function for the unit involute.
  // Note that the angles are all backwards (reversed sign) due to LH coords.
  
  let basicInv = FPath.parametricToBezier(invo2,0,1,30);
  let basicArc = FPath.circArcToBezier(rb,0,Math.PI/N - 2*alpha);
  basicArc = basicArc.reflectX();

  let angle = -Math.PI / (2*N) + alpha;
  ctx.strokeStyle='black';

  for (let i = 0; i < N; i++)
    {
      // One side of tooth.
      let inv = basicInv.rotate(angle);
      inv = inv.scale(rb);
      inv = inv.translate(c);
      ctx.stroke(inv);

      // Other side of tooth.
      angle -= Math.PI / N + 2 * alpha;
      inv = basicInv.reflectX();
      inv = inv.rotate(angle);
      inv = inv.scale(rb);
      inv = inv.translate(c);
      ctx.stroke(inv);

      // Gap between teeth.
      let gap = basicArc.rotate(angle);
      gap = gap.translate(c);
      //ctx.strokeStyle='blue';
      ctx.stroke(gap);

      angle -= Math.PI / N - 2*alpha;
    }    
}

\end{figput}
\caption{Basic Gear Form.}
\label{fig-gear-example}
\end{figure}

There is a glaring problem with Figure (\ref{fig-gear-example}): the
involutes continue beyond the point where the two sides of each tooth
meet. If the aim is to program a milling machine to cut these profiles,
then that's not a big deal -- the machine will be cutting a bit of
air beyond the end of each tooth -- but it would be nice to know
exactly where the two sides meet. Suppose that a tooth is symmetric
about the $x$-axis so that the two sides meet at $y=0$. Let $R_\theta$
be the rotation matrix through angle $\theta$. Then the involute below
the $x$-axis is parameterized by $R_{-\theta}i(t)$, where $\theta
= \alpha + \pi/2N$. In particular, we want to find $t$ such that the
$y$-coordinate of $R_{-\theta}i(t)$ is equal to zero. We have
\begin{eqnarray*}
R_{-\theta}\ i(t) &=& \pmatrix{\cos\theta&\sin\theta\cr-\sin\theta&\cos\theta}
\pmatrix{r_b(\cos t + t\sin t)\cr r_b(\sin t - t\cos t)},
\end{eqnarray*}
and we require $t$ such that
$$-r_b\sin\theta(\cos t + t\sin t) + r_b\cos\theta(\sin t - t\cos t) = 0$$
or
$${\sin t - t\cos t\over \cos t + t\sin t} = \tan\theta.$$
Unfortunately, finding such $t$ requires the use of numerical methods
of approximation. Making use of something like Newton-Raphson to
determine $t$, we obtain Figure (\ref{fig-gear-ex-better}).

\begin{figure}
\begin{figput}{bettergear,120bp}[done,skip]
function bettergear(ctx) {

  // As above, but the teeth meet properly instead of having ``hair.''
  
  // Pressure angle, number of teeth and module.
  let phi = 20 * Math.PI / 180;
  let N = 15;
  let m = 5;

  let c = new Point2D(175,60);
  let pitchDiam = N * m;

  // Pitch radius and base radius.
  let rp = pitchDiam / 2;
  let rb = rp * Math.cos(phi);

  ctx.lineWidth = 0.4;
  
  // Point along parameterization where involute meets pitch circle.
  let t = Math.tan(phi);

  // Adjustment to width of the base of the teeth due to the fact that
  // the teeth are of varying width. Note that the radius of the base circle
  // doesn't matter.
  let ipt = invo2(t);
  let alpha = -Math.atan2(ipt.y,ipt.x);

  ctx.lineWidth = 0.4;

  // Each tooth, at the base, takes 2pi/2N + 2 alpha. The space between
  // teeth is 2pi/2N - 2 alpha
  //
  // We want the x-axis to split a gap between teeth, so the first tooth
  // starts at (pi/N - 2 alpha) / 2 = pi/2N - alpha.
  //
  // Using earlier function for the unit involute.
  // Note that the angles are all backwards (reversed sign) due to LH coords.
  //
  // We also solve for t so that the teeth meet as they should.
  let targetV = Math.tan(Math.PI/(2*N) + alpha);
  let toothT = Numerical.newton(toothMeet,0.3,0,2,targetV,0.00001);
  
  let basicInv = FPath.parametricToBezier(invo2,0,toothT,30);
  let basicArc = FPath.circArcToBezier(rb,0,Math.PI/N - 2*alpha);
  basicArc = basicArc.reflectX();

  let angle = -Math.PI / (2*N) + alpha;
  ctx.strokeStyle='black';

  for (let i = 0; i < N; i++)
    {
      // One side of tooth.
      let inv = basicInv.rotate(angle);
      inv = inv.scale(rb);
      inv = inv.translate(c);
      ctx.stroke(inv);

      // Other side of tooth.
      angle -= Math.PI / N + 2 * alpha;
      inv = basicInv.reflectX();
      inv = inv.rotate(angle);
      inv = inv.scale(rb);
      inv = inv.translate(c);
      ctx.stroke(inv);

      // Gap between teeth.
      let gap = basicArc.rotate(angle);
      gap = gap.translate(c);
      //ctx.strokeStyle='blue';
      ctx.stroke(gap);

      angle -= Math.PI / N - 2*alpha;
    }
}

function toothMeet(t) {

  // Used to solve for the t at which the two sides of a tooth meet.
  let s = Math.sin(t);
  let c = Math.cos(t);
  return (s - t*c) / (c + t * s);
}

\end{figput}
\caption{Corrected Gear Form.}
\label{fig-gear-ex-better}
\end{figure}

Figure (\ref{fig-gear-ex-better}) still doesn't look quite
right. Gears don't typically have such pointy-ended teeth, and the gaps
between the teeth don't seem deep enough in Figure
(\ref{fig-gear-ex-better}). There are two more parameters to adjust
for this: the \emph{addendum}, $a$, and \emph{dedendum}, $b$. The
addendum is the distance above the pitch circle to which the teeth
extend; when a tooth reaches a radius of $r_p + a$, it is truncated
and given a flat top (the so-called \emph{top land}). The dedendum is
the depth below the pitch circle to which the gap between teeth is
cut; so the gaps are cut to a radius of $r_p -b$ (forming the
so-called \emph{bottom land}). While the parameters $a$ and $b$ could
take any value, they have been standardized to
$$a = m\qquad{\rm and}\qquad b = 1.25\ m.$$
Under the AGMA system (inches), these are
$$a = 1/p_d\qquad{\rm and}\qquad b = 1.25/p_d.$$
Teeth have been standardized this way because the tips of pointy-ended
teeth are prone to burring, while cutting the gaps more deeply allows
for fuller engagement of the teeth. Figure (\ref{fig-gear-with-add-ded})
shows the same gear as in Figure (\ref{fig-gear-ex-better}), but with
the addendum and dedendum circles.

\begin{figure}[b!]
\begin{figput}{gearaddded,120bp}[done,skip]
function gearaddded(ctx) {

  // As above, but with (or without when interactive) the various circles.
  
  // A button for show/hide the extra circles.
  let wbut = ButtonWidget.register(ctx,-80,10,'Show/Hide','but');
  
  // Pressure angle, number of teeth and module.
  let phi = 20 * Math.PI / 180;
  let N = 15;
  let m = 5;

  let c = new Point2D(175,60);
  let pitchDiam = N * m;

  // Pitch radius and base radius.
  let rp = pitchDiam / 2;
  let rb = rp * Math.cos(phi);

  ctx.lineWidth = 0.4;

  if (wbut.clickState === true)
    {
      // Pitch circle.
      p = new FPath();
      p.ellipse(c.x,c.y,rp,rp,0,0,2*Math.PI);
      ctx.stroke(p);

      // Addendum
      radd = rp + m;
      p = new FPath();
      p.ellipse(c.x,c.y,radd,radd,0,0,2*Math.PI);
      ctx.stroke(p);

      // Dedendum
      rded = rp - 1.25* m;
      p = new FPath();
      p.ellipse(c.x,c.y,rded,rded,0,0,2*Math.PI);
      ctx.stroke(p);
    }


  // Point along parameterization where involute meets pitch circle.
  let t = Math.tan(phi);

  // Adjustment to width of the base of the teeth due to the fact that
  // the teeth are of varying width. Note that the radius of the base circle
  // doesn't matter.
  let ipt = invo2(t);
  let alpha = -Math.atan2(ipt.y,ipt.x);

  ctx.lineWidth = 0.4;

  // Each tooth, at the base, takes 2pi/2N + 2 alpha. The space between
  // teeth is 2pi/2N - 2 alpha
  //
  // We want the x-axis to split a gap between teeth, so the first tooth
  // starts at (pi/N - 2 alpha) / 2 = pi/2N - alpha.
  //
  // Using earlier function for the unit involute.
  // Note that the angles are all backwards (reversed sign) due to LH coords.
  //
  // We also solve for t so that the teeth meet as they should.
  let targetV = Math.tan(Math.PI/(2*N) + alpha);
  let toothT = Numerical.newton(toothMeet,0.3,0,2,targetV,0.00001);
  
  let basicInv = FPath.parametricToBezier(invo2,0,toothT,30);
  let basicArc = FPath.circArcToBezier(rb,0,Math.PI/N - 2*alpha);
  basicArc = basicArc.reflectX();

  let angle = -Math.PI / (2*N) + alpha;
  ctx.strokeStyle='black';

  for (let i = 0; i < N; i++)
    {
      // One side of tooth.
      let inv = basicInv.rotate(angle);
      inv = inv.scale(rb);
      inv = inv.translate(c);
      ctx.stroke(inv);

      // Other side of tooth.
      angle -= Math.PI / N + 2 * alpha;
      inv = basicInv.reflectX();
      inv = inv.rotate(angle);
      inv = inv.scale(rb);
      inv = inv.translate(c);
      ctx.stroke(inv);

      // Gap between teeth.
      let gap = basicArc.rotate(angle);
      gap = gap.translate(c);
      //ctx.strokeStyle='blue';
      ctx.stroke(gap);

      angle -= Math.PI / N - 2*alpha;
    }
}

\end{figput}
\caption{Gear with Addedenum and Dedendum Circles.}
\label{fig-gear-with-add-ded}
\end{figure}

It is now possible to specify the standard tooth profile for a gear
with arbitrary parameters, as in Figure ({\ref{fig-final-gear}). The
value for $t$ to which the parameterization of the involute extends
must be adjusted. Instead of extending out to the value of $t_0$ for
which 
$${\sin t_0 - t_0\cos t_0\over \cos t_0 + t_0\sin t_0} = \tan\theta,$$
$t$ must be chosen so that $|i(t)| = r_p + a$. This is simpler to
determine since $\theta$ no longer plays a role. As in an earlier
calculation, we must have
\begin{eqnarray*}
(r_p+a)^2 &=& |i(t)|^2 \\
&=& r_b^2(1+t^2)
\end{eqnarray*}
or
$$t\ =\ \sqrt{\left({r_p+a\over r_b}\right)^2 - 1}\ =\ 
\sqrt{\left({r_p+a\over r_p\cos\phi}\right)^2 - 1}.$$
Of course, $a$ can't be chosen to produce a value for $t$ larger than $t_0$.

When drawing gears with a given addendum, it can be useful to know the
angle subtended by the top land. As noted above, the angle subtended
relative to the base circle by each tooth is $\pi/N + 2\alpha$. Let
$t_d$ be the value for $t$ at which the top land begins. Each
side of the tooth, from the base circle to the top land subtends the
angle $\beta$, where $\tan\beta = i_y(t_d)/i_x(t_d)$. The top land
thus subtends the angle $\pi/N + 2\alpha - 2\beta$.

\begin{figure}
\begin{figput}{gearfinal,200bp}[done,skip]
function gearfinal(ctx) {

  // As above, but allow the user to tweak the settings. Further, we use the
  // standard values for addendum and dedendum. This changes the way the
  // involutes and gaps are drawn.
  //
  // BUG: There's some problem here of a mathematical or graphical nature.
  // If I lift the restrictions on the inputs, things go haywire at
  // the extremes.
  
  // Default pressure angle, number of teeth and module.
  let phi = 20 * Math.PI / 180;
  let N = 15;
  let m = 5;

  // Let the user change these.
  let leftInput = -50;
  let toothInput = NumberInputWidget.register(ctx,leftInput,20,15,"tooth");
  N = parseInt(toothInput.getValue());
  if (N < 4)
    {
      toothInput.theWidget.value = 4;
      N = 4;
    }
  if (N > 30)
    {
      toothInput.theWidget.value = 30;
      N = 30;
    }
    
  let paInput = NumberInputWidget.register(ctx,leftInput,35,20,"pa");
  phi = parseFloat(paInput.getValue());
  if (phi > 25)
    {
      paInput.theWidget.value = 25;
      phi = 25;
    }      
  if (phi < 5)
    {
      paInput.theWidget.value = 5;
      phi = 5;
    }
  phi = phi * Math.PI / 180;

  let moduleInput = NumberInputWidget.register(ctx,leftInput,50,5,"mod");
  m = parseFloat(moduleInput.getValue());
  if (m > 6)
    {
      moduleInput.theWidget.value = 6;
      m = 6;
    }
  if (m < 1)
    {
      moduleInput.theWidget.value = 1;
      m = 1;
    }

  let leftText = -120;
  let upText = 30;
  ctx.font = '10px san-serif';
  ctx.fillStyle = 'black';
  drawTextBrowserOnly(ctx,'Tooth Count',leftText,35-upText);
  drawTextBrowserOnly(ctx,'Pressure Angle',leftText,50-upText);
  drawTextBrowserOnly(ctx,'Module',leftText,65-upText);
  
  // Addendum and dedendum.
  // BUG: I could let the user adjust these too.
  let a = m;
  let b = 1.25 * m;
  
  let c = new Point2D(150,100);
  let pitchDiam = N * m;

  // Pitch radius and base radius.
  let rp = pitchDiam / 2;
  let rb = rp * Math.cos(phi);

  ctx.lineWidth = 0.4;

  // Center of gear.
  let p = new FPath();
  p.ellipse(c.x,c.y,1.5,1.5,0,0,2*Math.PI);
  ctx.fill(p);

  // Point along parameterization where involute meets pitch circle.
  let t = Math.tan(phi);

  // Adjustment to width of the base of the teeth due to the fact that
  // the teeth are of varying width. Note that the radius of the base circle
  // doesn't matter.
  let ipt = invo2(t);
  let alpha = -Math.atan2(ipt.y,ipt.x);

  // Determine the maximum value for t. This is the point at which the
  // sides of a tooth meet. It determines the maximum possible value
  // for the addendum.
  let targetV = Math.tan(Math.PI/(2*N) + alpha);
  let maxT = Numerical.newton(toothMeet,0.3,0,2,targetV,0.00001);

  // Now the value of t determined by the addendum.
  let T = Math.sqrt(((rp + a)/rb)**2 - 1);
  if (T > maxT)
    T = maxT;

  // These are the parts of the gear profile. There's a tricky thing here in
  // that the arc subtended by the top land must be determined.
  // The arc subtended by half a tooth (out to maxT) is known to be
  // pi/(2N) + alpha, but the involute now spans a smaller angle and the top
  // land makes up the difference. This smaller angle is determined by
  // invo(T), and the angle is beta such that
  // tan(beta) = y(T)/x(T).
  let basicInv = FPath.parametricToBezier(invo2,0,T,30);
  let basicBot = FPath.circArcToBezier(rp-b,0,Math.PI/N - 2*alpha);

  ipt = invo2(T);
  let beta = -Math.atan2(ipt.y,ipt.x);
  let basicTop = FPath.circArcToBezier(rp+a,0,Math.PI/N + 2*alpha - 2*beta);

  // Due to LH coordinates.
  basicTop = basicTop.reflectX();
  basicBot = basicBot.reflectX();

  // Add a radial line on each side of basicInv so that the tooth reaches
  // the dedendum circle.
  basicInv = basicInv.scale(rb);
  basicInv.frontLineTo(rp-b,0);

  let angle = -Math.PI / (2*N) + alpha;
  ctx.strokeStyle='black';

  for (let i = 0; i < N; i++)
    {
      // One side of tooth.
      let inv = basicInv.rotate(angle);
      inv = inv.translate(c);
      ctx.stroke(inv);

      // Arc of top land.
      let top = basicTop.rotate(angle - beta);
      top = top.translate(c);
      ctx.stroke(top);
      
      // Other side of tooth.
      angle -= Math.PI / N + 2 * alpha;
      inv = basicInv.reflectX();
      inv = inv.rotate(angle);
      inv = inv.translate(c);
      ctx.stroke(inv);

      // Gap between teeth.
      let gap = basicBot.rotate(angle);
      gap = gap.translate(c);
      ctx.stroke(gap);

      angle -= Math.PI / N - 2*alpha;
    }
}
\end{figput}
\caption{Standard Gear Profile.}
\label{fig-final-gear}
\end{figure}
















\end{document}


